\chapter{Problemanalyse}
\label{chap:analyse}
Betrachtet man die Ausgangssituation aus Kapitel \secref{sec:ausgangssituation} konnte sich das insolvente Unternehmen nicht auf dem Markt durchsetzten. Zudem konnten die Kosten für Wartung und Weiterentwicklung nicht mehr getragen werden. Die Time-to-Market (TTM) Zeitspanne war zu groß um den Break-even-Point zu erreichen. Zudem kann es passieren, dass in dieser Zeit neue Technologien von Konkurrenz Unternehmen veröffentlicht werden, wodurch das eigene Produkt nicht mehr Erfolgreich vermarktet werden kann, sobald dieses Fertiggestellt worden ist. Dies war ein weiterer Grund, warum Dienste wie Amazon die eigenen Produkte vom Markt gedrängt haben.

\section{Herausforderungen}
\label{sec:herausforderung}
Oft passiert es, das ein Unternehmen große Vorstellungen von dem Unternehmens-Ziel hat. Dies führt häufig dazu, dass Software entwickelt wird, welches weit über den aktuellen Anforderungen hinaus gehen. Man möchte damit verhindern, Software an einem späteren Zeitpunkt neu zu entwickeln oder austauschen zu müssen. Jedoch kann dies zu großen Problemen führen sobald das Unternehmen wächst und den am Anfang genannten Vorstellungen näher kommt. In dieser Phase entstehen meistens Probleme, welche vorher nicht berücksichtigt worden sind, weil niemand sie kannte. Dadurch muss die vorhandene Software, welche ursprünglich für dieses Szenario ausgelegt war, geändert werden.
\\\\
Eine weitere Herausforderung besteht in der Umstellung auf eine Service-orientierte Architektur. Wurde, wie im Fall von der \textit{Auktionen GmbH}, zunächst auf ein monolithisches System gesetzt, ist die Umstellung auf ein Service-orientierte Architektur nicht ganz trivial. Die bestehende Anwendung muss weiter entwickelt werden, während auf die neue Architektur umgestellt wird.

\section{Die zu untersuchende Fragestellung}
\label{sec:dasProblem}
Das Problem besteht darin, eine Anwendung flexibel und einfach erweiterbar zu gestalten, damit ein Unternehmen schnell auf Änderungen und die veränderten Bedürfnisse der Nutzer reagieren kann. Damit solch eine Software ebenfalls gut Wartbar ist, sollten Redundanzen möglichst vermieden werden. Das heißt eine Funktionalität ist nur einmal im gesamten Unternehmen vorhanden. So können zum Beispiel alle Codestücke einer bestimmten Berechnung in ein Modul gepackt werden, wodurch diese  nur an einer zentralen Stellen geändert werden muss.
\\\\
Um dieses Vorgehen zu vereinfachen hat man sich dazu entschieden, diese Codestücke in eigene Applikationen (Dienste) zu verpacken und diese über eine Schnittstelle anzubieten. Das sorgt dafür, dass jede Software, welche dieses Modul benötigt, sie nicht mehr eigenständig implementieren muss, sondern den dafür vorgesehenen Dienst aufrufe kann, um die nötigen Informationen zu erhalten. Hierbei spricht man von einer Service-orientierten Architektur. Sowohl SOA als auch Microservice kommen in diesem Szenario in frage.Es stellen sich dementsprechend folgende Fragen:
\begin{enumerate}
    \item Wo liegen die Unterschiede zwischen diesen beiden Modellen?
    \item Welche Vor- und Nachteile hat das jeweilige Modell?
    \item Gibt es Grenzen oder Beschränkungen bei der Benutzung eines Modelles?
    \item In wie weit helfen die Architekturmodelle die angesprochenen Problematiken aus Kapitel \secref{sec:motivation}\ zu lösen?
\end{enumerate}
     