\chapter{Microservices}
\label{chap:Microservices}

\section{Überblick}
\label{sec:überblickMicroservice}
\begin{quotation}
    \frqq Modularisierung ist nichts Neues. Schon lange werden große Systeme in kleine Module unterteilt, um Software einfacher zu erstellen, zu verstehen und weiterzuentwickeln. Das Neue: Microservices nutzen als Module einzelne Programme, die als eigene Prozesse laufen. Der Ansatz basiert auf der UNIX-Philosophie. Sie lässt sich auf drei Aspekte reduzieren:\flqq\cite[S. 2]{EWolff2016:Microservices}
\end{quotation}


\begin{itemize}
    \item Ein Programm soll nur eine Aufgabe erledigen, und das soll es gut machen.
    \item Programme sollen zusammenarbeiten können.
    \item Nutze eine universelle Schnittstelle. In UNIX sind das Textströme.
\end{itemize}
Diese Art der Aufteilung wurde schon lange von großen Unternehmen wie Amazon oder Google genutzt und wurde zu nächst \SOA (SOA) genannt. Jedoch unterscheiden sich Microservices und SOA voneinander. Daher wird SOA im Nächsten Kapitel genauer erläutert und im Kapitel \ref{chap:ergebnisse} die Ergebnisse zusammen gefasst und beide Technologien mit einander verglichen.
\\\\
Der Begriff Microservices ist nicht eindeutig definiert. Als erste Näherung dienen, nach Eberhard Wolff \cite[S. 2]{EWolff2016:Microservices}, folgende Kriterien:

\begin{itemize}
    \item Microservices sind ein Modularisierungskonzept. Sie dienen dazu. ein großes Software-System aufzuteilen - und beeinflussen die Organisation und die Software-Entwicklungsprozesse.
    \item Microservices können unabhängig von Änderungen an anderen Microservices in Produktion gebracht werden.
    \item Microservices können in unterschiedlichen Technologien implementiert sein. Es gibt keine Einschränkung auf eine bestimmte Programmiersprache oder Plattform.
    \item Microservices haben einen eigenen Datenhaushalt: eine eigene Datenbank - oder ein vollständig getrenntes Schema in einer gemeinsamen Datenbank.
    \item Microservices können eigene Unterstützungsdienste mitbringen, beispielsweise eine Suchmaschine oder eine spezielle Datenbank. Natürlich gibt es eine gemeinsame Basis für alle Microservices - beispielsweise die Ausführung virtueller Maschinen.
    \item Microservices sind eigenständige Prozesse - oder virtuelle Maschinen, um auch die Unterstützungsdienste mitzubringen.
    \item Dementsprechend müssen Microservices über das Netzwerk kommunizieren. Dazu nutzen Microservices Protokolle, die lose Kopplung unterstützen. Das kann beispielsweise REST sein - oder Messaging-Lösungen.
\end{itemize}

Grundsätzlich kann man Microservices in drei Kategorien einteilen:
\begin{description}
    \item[Producer] Ein Service der etwas Produziert oder auf eine Anfrage reagiert. Das reicht von Daten aus einer Datenbank extrahieren bis hin zu komplexen Berechnungen.
    \item[Consumer] Ein Service oder eine Anwendung, welche einen oder mehrere Produzierende Services verwendet und entweder weiterverarbeitet oder ausgibt. Im Falle der Weiterverarbeitung ist ein Consumer ebenfalls ein Producer sein.
    \item[Self-Contained System (SCS)] "\frqq Microservice mit UI\flqq\ oder \frqq Self-Contained System\flqq\ wie es Stefan Tilkov nennt, sind in sich abgeschlossene Systeme. [..] Sie enthalten eine UI und sollten möglichst nicht mit anderen SCS kommunizieren." \cite[vgl S. 55]{EWolff2016:Microservices}. SCS sind jedoch nur im Microservice und nicht im SOA Umfeld zu finden. Warum wird in \secref{chap:fallstudie} weiter erläutert.
\end{description}

% Eigene Datenbanken

\section{Gr"o\ss e von Microservices}
\label{sec:groesseMicroservice}
\begin{quotation}
    \frqq Der Name \frqq Microservices\flqq\ verrät schon, dass es um die Servicegröße geht - offensichtlich sollen die Services klein sein."\cite[S. 31]{EWolff2016:Microservices} Es gibt verschiedene Möglichkeiten die Größe von Programmen zu ermitteln. Eine Variante ist zum Beispiel das Zählen von  Lines of Code (LOC), jedoch hat diese Methode auch Nachteile. Denn die Anzahl der Codezeilen hängen stark von der verwendeten Programmiersprache ab. Einige Programmiersprachen benötigen mehr Zeilen Code, um eine bestimmte Tätigkeit abzubilden, als andere.
    \\
    Die Größe von Services sollte jedoch nicht von zentraler Bedeutung sein, denn eine untere Grenze gibt es für Services nicht. "Wohl aber eine obere Grenze: Wenn der Microservice so groß ist, dass er von einem Team nicht mehr weiterentwickelt werden kann, ist sie zu groß. Ein Team sollte dabei eine Größe haben, wie sie für agile Prozesse besonders gut funktioniert. Das sind typischerweise drei bis neun Personen.\flqq\cite[S. 34]{EWolff2016:Microservices}
\end{quotation}

Bei der Größe eines Services ist jedoch darauf zu achten, das ein Service nicht zu viele oder zu wenige Funktionen besitzt. Wie bereits beschrieben, sind Microservices modulare, lose gekoppelte Services. Wird ein Service zu klein angesetzt, können daraus Abhängigkeiten zu anderen Services entstehen und damit das gesetzt der losen Kopplung verletzten. Besitzt hingegen ein Service zu viele Funktionen, wird es meistens nicht mehr als Microservice angesehen, da es nicht eine, sondern mehrere Aufgaben übernimmt und diese wahrscheinlich nicht mehr gut erledigen kann.

\section{Orchestration vs Choreographie}
\label{sec:orchestrationvschoreographie}
Möchte man ein Microservice System aufbauen, stellt sich die Frage, wie einzelne Services Strukturiert werden und wie diese unter einander kommunizieren sollen. Ein bestimmter Vorgang startet in der Regel bei einem Service. Nun muss man entscheiden ob weitere Services hinzugezogen, beziehungsweise informiert werden müssen.
Je nach Anwendungsfall muss man sich zwischen Service Orchestration und Choreographie entscheiden. Dabei ist es fast unmöglich ein ganzes Microservice-System aus nur einem der beiden Varianten zu bauen.

% [Seite 66 EWolff2016] Technologische Wahlfreiheit
\subsection{Herausforderung}
\label{sec:Herausforderung}
Der Ausfall eines Services kann im schlechtesten Fall dazu führen, dass alle anderen Microservices nicht mehr funktionieren. Um das zu verhindern muss klar definiert werden, was Microservices in dieser Situation tun sollen. Zusätzlich muss, sofern Datenbank Operationen eine wichtige Rolle Spielen, das Problem der einheitlichen Transaktion gelöst werden. Wenn zum Beispiel eine Operation Daten über verschiedene Microservices in Datenbanken schreibt, muss bei nicht Erreichbarkeit oder Fehlers eines Microservices eine einheitlicher Rollback durchgeführt werden, um keine inkonsistente Dateien im System zu haben.
Eine weitere Herausforderung besteht in dem Grundkonzept von Microservices. Da nicht definiert ist, welche Programmiersprache für Microservices verwendet wird, kann ein Service zum Beispiel in Java, ein anderes in Scala ode Python geschrieben werden. Es muss daher dafür gesorgt werden, dass die einzelnen Services untereinander interoperabel sind. Um das zu gewährleisten, müssen die Schnittstellen möglichst einheitlich und auf dem gleichen Protokoll aufbauend programmiert werden. Hier bieten REST-Schnittstellen eine gute Lösung. Diese können auf dem HTTP-Protokoll aufgebaut werden. Zusätzlich bietet das HTTP-Protokoll die Möglichkeit, ein einheitliches Medium, wie zum Beispiel XML oder JSON, als Informationsträger zu nutzen. Dabei kann theoretisch jeder Microservice mit jedem anderen Microservice kommunizieren, sofern die Schnittstellen einheitlich definiert sind.

\section{PUSH- VS PULL-Architektur}
\label{sec:PushPullArchitektur}
Grundlegend können Microservices mit Hilfe zwei verschiedener Kommunikations-Architekturen kommunizieren, PUSH- und PULL-Architektur. Dabei ist jedoch nicht ausgeschlossen, dass sobald eine Architektur gewählt worden ist, die andere nicht mehr genutzt werden kann. Genauso wie bei der Entscheidung über die Kommunikationsstruktur (siehe \secref{subsec:orchestration}), kann es von Vorteil sein, beide Architekturen zu nutzen.
\\\\
\textbf{PULL-Architektur}\\
Eine PULL-Architektur basiert auf einen einfachen Request-Replay-Schema. Dementsprechend ist das Web PULL-basiert.
Der Browser macht eine Anfrage an einen Server, dieser wiederum verarbeitet die Anfrage und liefert eine Antwort (Replay) zurück. Dies hat den Vorteil, das nicht lange auf eine Antwort gewartet werden muss und die teilhabenden Kommunikationspartner gegenseitig kennen, jedoch bringt es auch den Nachteil, dass dadurch weitgehend eine synchrone Kommunikation stattfindet und eine Antwort häufig nicht gleichzeitig an mehrere Empfänger senden kann.
\\\\
\textbf{PUSH-Architektur}\\
Eine PUSH-Architektur wird eingesetzt, wenn man verschiedene Kommunikationspartner über bestimmte Ereignisse informieren möchte. Hier stehen meist nicht die Kommunikationspartner, sondern die Informationen im Vordergrund. Dafür wird meistens ein eigenständiger Service (Broadcaster) eingesetzt, der die Verteilung dieser Informationen übernimmt. Dabei kann ein Service als Informationsprovider dienen, zum Beispiel ein Nachrichten-Feed (Von einer Nachrichtenseite). Alle anderen Services abonnieren den Broadcaster und erhalten dadurch alle Nachrichten, die der Informationsprovider sendet.
Es gibt jedoch auch den Fall, dass die Kommunikation sternförmig um den Broadcaster angeordnet sind. Dadurch ist jeder Service der diesen abonniert, sowohl Provider, als auch Consumer.
Anders als bei PULL-basierten Systemen kann hier nicht unbedingt sichergestellt werden, dass alle Nachrichten von allen Konsumern gleichzeitig gelesen und ggf. verarbeitet werden. Jedoch können so Informationen innerhalb eines Microservice-Systems relativ zuverlässig verteilt werden.
Der Vortiel von PUSH-Architekturen ist, dass eine asynchrone Informationsverbreitung aufgebaut werden kann. Zudem können Serviceausfälle, solange es nicht der Boradcaster oder wichtige Microservices sind, überbrückt werden, indem der Broadcaster die Nachrichten für eine bestimmte Zeit vorhält und so der Microservice, welcher nicht erreichbar war, die Nachrichten trotzdem noch erhält.
\\\\
Oft ist es nicht notwendig eine Antwort zu erhalten. Zum Beispiel muss eine Registrierung in unserem fiktiven Unternehmen, der \gmbh\ möglich sein. Dabei sendet der Microservice der für die Registrierung zuständig ist eine einfache Event-Nachricht, wie Benutzer XY hat sich Registriert. Im Hintergrund kann dann zum Beispiel ein anderer Microservice diese Nachricht erhalten und zusätzliche Aktionen durchführen, wie erstellen des Warenkorbs.