\chapter{Vergleich}
\label{chap:vergleich}
Nachdem erläutert wurde, was unter \textbf{Microservice} und \textbf{Service-orientierte Architektur (SOA)} zu verstehen ist, werden nun beide Architektur Modelle mit einander verglichen und kritisch betrachtet. Anschließend wird aus dem Vergleich ein Fazit gezogen und dieses erläutert.
\\\\
Bevor die Architekturmodelle mit einander verglichen werden können, muss erwähnt werden, dass ein direkter Vergleich zwischen Funktionen und den Vor- und Nachteilen der jeweiligen Modelle nicht möglich ist. Außerdem ist weder der Begriff Microservice, noch der Begriff SOA eindeutig definiert. Literaturen wie \cite{100QA}\ und \cite{EWolff2016:Microservices}\ bieten einen guten Ansatz, um diese beiden Begriffe grob einzuordnen.

\section{Grundlagen}
\label{sec:FazitGrundlagen}
Der Grundgedanke beider Architekturen ist sehr unterschiedlich. Zwar stammen beide Modellarten aus dem Bereich der "`Service orientierten Architekturen"' und auch \textit{SOA} ist eine Abkürzung dieses Begriffes, jedoch verfolgen beide Modelle verschiedene Ansätze.  
\\\\
SOA hat seine Grundlagen nicht in der IT, sondern in der Geschäftswelt. Dies wird dadurch bestätigt, dass die Zielgruppe der Befragten in Kapitel \ref{chap:soa}\ \nameref{chap:soa}\ aus dem kaufmännischen und analytischen Bereich stammen. So wird zum Beispiel der \flqq business executive\frqq\ und der \flqq business analyst\frqq\ gefragt, wie SOA zu verstehen ist. SOA ist demnach \textit{business-driven}. Zudem darf SOA nicht als Modell, sondern als Herangehensweise verstanden werden.
\\\\
Microservices hingegen stammen aus der Notwendigkeit große Software möglichst effizient zu erstellen und die Wartung zu vereinfachen. Wie auch schon in Kapitel \ref{sec:überblickMicroservice} \nameref{sec:überblickMicroservice} erwähnt, ist Modularisierung nichts neues und wird aus den gerade genannten Notwendigkeiten eingesetzt. Die Microservice-Architektur ist nichts anderes als die Modularisierung einer Software, bei der diese in kleine Software Pakete, sogenannte Services, geteilt werden.
\\\\
Während bei SOA das Geschäft und die Geschäftsprozesse im Vordergrund steht, will man mit Microservices die Entwicklung von Software unterstützen. Das Microservice-Modell wurde entwickelt, damit umfangreiche Software möglichst einfach und schnell mit vielen Personen entwickelt werden kann. Bei SOA geht es um den möglichst effizienten Einsatz von Software und nicht deren Entwicklung. Oftmals existiert schon Unternehmenssoftware. Durch SOA soll diese möglichst effizient untereinander kommunizieren.
\\\\
Grundsätzlich wurden schon viel früher Architektur-Modelle entwickelt, um die in den jeweiligen Kapiteln \ref{chap:soa} \nameref{chap:soa} und \ref{chap:Microservices} \nameref{chap:Microservices} genannten Probleme und Anforderungen umzusetzen. Ein Begriff für diese Modelle existierte jedoch noch nicht. Die jeweiligen Begriffe wurden erst später verwendet, um die jeweiligen Modelle spezifizieren und erklären zu können.

\section{Architektur}
\label{sec:FazitArchitektur}
Wie bereits weiter oben erläutert, wird mit SOA versucht verschiedene Geschäftsprozesse miteinander zu verknüpfen und eine leichtere Kommunikation unter diesen zu ermöglichen. Dadurch sind die meisten Dienste nicht  Bestandteil einer großen Software, sondern nur Bestandteil eines Unternehmensbereiches. Oft wird daher versucht alle EDV-Komponenten miteinander zu verknüpfen ohne eine Abhängigkeit zu erschaffen. Die Anwendungen selber sind dabei keine Dienste. Damit diese ihre Funktionalität für andere Anwendungen bereitstellen kann, muss ein Adapter entwickelt werden. Dieser Adapter wird als Dienst bereitgestellt.
\\\\
Das Microservice-Modell arbeitet zwar auch mit Diensten, jedoch sind diese meist nicht so groß wie im SOA-Modell. Während bei dem SOA-Modell auch ganze Anwendungen mit Hilfe eines Adapters interoperable gemacht werden, wird mit dem Microservice-Modelle versucht eine ganze Anwendung in eigenständige  Dienste aufzuteilen. Dabei soll ein Dienst nur eine Aufgabe erledigen, diese aber jedoch besonders gut. Zudem gibt es in der Regel keine zentrale Anwendung, welche als Schnittstelle zwischen Datenbanken und Frontend dient.

\section{Kommunikation}
\label{sec:FazitKommunikation}
Nachdem die Architektur Unterschiede erläutert wurden, werden nun die Unterschiede in der Kommunikation erläutert.
\\\\
Die Modelle unterscheiden sich hinsichtlich der Anordnung von Diensten und damit der Kommunikationsfluss innerhalb der Modelle. Grundsätzlich kann man zwischen zwei Kommunikationsflüssen unterscheiden:
\begin{enumerate}
    \item \nameref{subsec:orchestration}
    \item \nameref{subsec:choreographie}
\end{enumerate}
Nicht beide Kommunikationsflüsse sind in beiden Service-orientierten Architekturen möglich. In SOA ist der ESB die zentrale Einheit über welches die gesamte Kommunikation läuft. Dadurch ist nur eine Orchestration der Dienste möglich und die direkte Kommunikation  zwischen den Diensten ist nicht möglich. Bei Microservices hingegen können beide Kommunikationsflüsse realisiert werden, wodurch die Kommunikation viel dynamischer gestaltet werden kann, da jeder Dienst mit jedem anderen Dienst kommunizieren kann. Oftmals sprechen Frontend Anwendungen wie Websiten direkt mit mehreren Services um bestimmte Informationen zu erhalten.

\section{Beteiligte Personen}
\label{sec:FazitBeteiligtePersonen}
Neben den technischen Unterschiede, gibt es auch Unterschiede hinsichtlich der Personengruppen die beteiligt sind. Dabei geht es nicht um die Nutzer einer Software, sondern um Personen die aktiv im Prozess der jeweiligen Modelle beteiligt sind.
\\\\
Bei dem SOA-Modell wurden weiter oben einige Personengruppen identifiziert. Unter anderem gehören darunter Business Executiver und Business Analysten. Weitere Personengruppen sind zum Beispiel Personen aus dem kaufmännischen Bereich. Natürlich spielen auch Personen aus der IT-Abteilung eine Rolle, wie zum Beispiel Administratoren und Architekten. Da SOA zum Teil auf bestehende Anwendungen aufbaut wird meistens nur ein Adapter benötigt, welcher die Anwendung SOA-fähig macht. Dadurch kann meistens der Entwicklungsanteil gering gehalten werden.
\\\\
Anders als beim SOA-Modell, steht beim Microservice-Modell die Entwicklung im Vordergrund. Wie bereits beschrieben, wird mit Microservices eine Anwendungen in viele eigenständige Dienste aufgeteilt. Hier spielen Entwickler eine zentrale Rolle, jedoch auch Personen aus dem kaufmännischen Bereich, da IT nie als selbst Zweck existieren darf, sondern immer die vorhandenen Businessprozesse unterstützen soll.
\\\\
Die Unterschiede in den beteiligten Personen zeigen neben den unterschiedlichen Personengruppen auch in welchem Umfeld sich beide Architekturmodelle bewegen bzw. aus welchem sie entstanden sind.