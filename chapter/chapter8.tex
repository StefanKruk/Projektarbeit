\chapter{Zusammenfassung und Ausblick}
\label{chap:ZusammenfassungUndAusblick}

\section{Zusammenfassung}
\label{sec:Zusammenfassung}
Ziel dieser Projektarbeit war es, die Unterschiede zwischen dem Microservice- und dem SOA-Paradigma herauszuarbeiten und zu bewerten. Dafür wurde zunächst eine Ausgangssituation dargelegt und darauf aufbauend eine Problemstellung formuliert. Diese wurde anhand eines hypothetischem Beispiels in Anlehnung an ein reales Problem verdeutlicht. Anschließend wurden die zentralen Fragestellungen formuliert.
\\\\
Damit diese Fragestellungen beantwortet werden können, wurden zunächst die allgemeinen Grundlagen zur Verwendung Service-orientierter Systeme erläutert. Darunter zählte unter anderem die Erläuterung der Architektur, sowie Domain-Driven Design und Bounded Context.
\\\\
Im Anschluss, wurde auf das SOA-Paradigma eingegangen. Dazu zählt unter anderem der Zusammenhang zwischen der Businessabteilung und der IT-Abteilung. Daran Anknüpfend, wurde das Microservice-Paradigma erläutert.
\\\\
Abschließend wurden beide Paradigmen, soweit es möglich war, miteinander verglichen und anschließend die zentralen Fragestellungen aus Kapitel \secref{sec:dasProblem}\ versucht zu beantworten, woraus sich das Fazit ableiten lies.

\section{Ausblick}
\label{sec:Ausblick}
Da in dieser Arbeit die Unterschiede zwischen dem Microservice-Paradigma und dem SOA-Paradigma ausgearbeitet wurden, kann darauf aufbauend zum Beispiel weiter auf eines der Paradigmen eingegangen werden. Es kann etwa für das Microservice-Paradigma eine Analyse erstellt werden, wie viel Aufwand es kostet, eine bestehende monolithische Anwendung, auf ein Microservice-System umzustellen.
\\\\
Für das SOA-Paradigma kann analysiert werden, wie viel Aufwand das Unternehmen aufwenden muss, um die Geschäftsprozesse in ein SOA-System zu überführen. Zusätzlich kann analysiert werden, welche nötigen Schritte für diese Umstellung nötig sind.