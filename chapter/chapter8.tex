\chapter{Zusammenfassung und Ausblick}
\label{chap:ZusammenfassungUndAusblick}

\section{Zusammenfassung}
\label{sec:Zusammenfassung}
Ziel dieser Projektarbeit war es, die Unterschiede zwischen dem Microservice-Modell und dem SOA-Modell herauszuarbeiten und zu bewerten. Dafür wurde zunächst eine Ausgangssituation dargelegt und darauf aufbauend eine Problemstellung formuliert, welches anhand eines hypothetischem Beispiels in Anlehnung an ein reales Problem verdeutlicht wurde. Anschließend wurden die Zentralen Fragestellungen formuliert.
\\\\
Damit die zentralen Fragestellungen beantwortet werden können, wurden zunächst die Allgemeinen Grundlagen zur Verwendung Service-orientierter Systeme erläutert. Darunter zählte unter anderem die Erläuterung der Architektur, sowie Domain-Driven Design und Bounded Context.
\\\\
Nachdem die Grundlagen erläutert wurden, wurde auf das SOA-Modell eingegangen. Dazu zählt unter anderem der Zusammenhang zwischen der Businessabteilung und der IT-Abteilung. Anschließend wurde das Microservice-Modell erläutert.
\\\\
Abschließend wurden beide Modelle, soweit es möglich war, miteinander verglichen und anschließend die zentralen Fragestellungen aus Kapitel \secref{sec:dasProblem}\ versucht zu beantworten, woraus sich das Fazit ableiten lies.

\section{Ausblick}
\label{sec:Ausblick}
Da in dieser Arbeit die Unterschiede zwischen dem Microservice-Modell und dem SOA-Modell ausgearbeitet wurden, kann darauf aufbauend zum Beispiel weiter auf eines der Modelle eingegangen werden. Außerdem könnte weitergehend beide Modelle in einem realen Beispiel implementiert und die Unterschiede dabei ausgearbeitet werden.