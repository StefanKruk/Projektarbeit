\chapter{Einleitung}
\label{chap:einleitung}

\section{Motivation}
In einem Unternehmen sind oft verschiedene Softwareprodukte im Einsatz. Damit ein reibungsloser Ablauf der internen Prozesse gewährleistet werden kann, sollte im Idealfall Software Interoperabel sein und so ohne weitere Probleme mit einander Arbeiten. Eine Lösung ist die \SOA. Dabei wird ein ganzer Geschäftsprozess in einer Monolithischen Anwendung, welcher zu einem bestimmten Kontext gehört, abgebildet. Eine andere Lösung sind Microservices. Diese Services sind, wie der Name schon sagt, in kleine und selbständige Anwendungen aufgeteilt, die nur eine bestimmte Aufgabe haben. Außerdem enthalten sie nur die reine Businesslogik. Damit Microservices für einen bestimmten Kontext genutzt werden können, müssen sie zusätzlich Orchestriert werden.

In dieser Arbeit sollen die Vor- und Nachteile beider Lösungen, unter anderem durch eine Implementierung, erörtert und danach miteinander verglichen werden. Abschließend werden die Ergebnisse präsentiert.

\section{Ausgangssituation}


\section{Vorgehen}