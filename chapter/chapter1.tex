\chapter{Einleitung}
\label{chap:einleitung}
Nur ein kurzes Beispiel, für den Umgang mit einem Glossar:
\\\\
In unserem Netzwerk setzen wir auf \gls{AD}. Durch den Einsatz
eines \gls{AD} erreichen wir bei \gls{MS}-Systemen, die mit einer
\gls{glos:AntwD} von \gls{CD} installiert wurden, die beste Standardisierung.

\section{Griechische Symbole}
Berechnungen mit \gls{symb:Pi} ergeben stets ein ungenaues Ergebnis,
denn \gls{symb:Pi} ist eine irrationale Zahl. Weiterhin gibt es noch
\gls{symb:Phi} und \gls{symb:Lambda}.

\section{Motivation}
{\bf WICHTIGER HINWEIS:} 
Alles in diesem {\bf Entwurf} den Aspekt
{\bf wissenschaftliches Arbeiten} bei der Anfertigung einer
Bachelor-/Master-Arbeit 
Betreffende ist absolut {\bf verbindlich} und muss uneingeschränkt
berücksichtigt werden! \\[0.1cm]
Alles in diesem {\bf Entwurf} das 
{\bf Layout} einer Bachelor-/Master-Arbeit Betreffende gibt die 
{\bf persönliche} Meinung des 
Verfassers wieder! Jede Studentin und jeder Student mag eigene Vorstellungen
entwickeln! Hier findet man eventuell erste hilfreiche Anhaltspunkte. \\[0.1cm]
Nun zum Überblick! Was sollte er u.\ a.\ enthalten:
\begin{itemize}
\itemsep -6pt
\item Erläuterung der Problemstellung 
\item Motivation für die Beschäftigung mit dem Problem
\item Hinweis auf eventuell schon vorhandene Entwicklungen 
\item Abgrenzung der eigenen Arbeit von eventuell schon vorhandenen
      Entwicklungen 
\item kurzer Abriss über den Inhalt der Arbeit
\end{itemize}

\section{Notation}
In diesem Abschnitt wird die in dieser Arbeit verwendete Notation 
vorgestellt und erläutert, falls sie relativ aufwendig ist.

\section{Grundlagen}
Wohlbekannte, aber für diese Arbeit besonders wichtige Resultate finden
sich hier, wenn sie denn benötigt werden. Wichtig in diesem und in allen
folgenden Kapiteln: {\bf Zitate angeben!!!} Wann immer etwas aus einem Buch,
einer Veröffentlichung, einem Vortrag oder einer www-Page entnommen ist, 
{\bf muss} dies durch eine entsprechende Angabe der Quelle im Text (z.B.
\cite[vgl.][S.\ 23ff]{Lenze_Einfuehrung_2000}) sowie einer Angabe der vollständigen Quelle im
Literaturverzeichnis kenntlich gemacht werden! Es ist absolut unzulässig,
längere Passagen {\bf wörtlich} oder {\bf sinngemäß und nahezu wörtlich} aus
einem anderen Dokument zu übernehmen, ohne dies präzise zu zitieren,
auch wenn man die Referenz pauschal im Literaturverzeichnis angibt
(Plagiat, nicht bestanden, keine Wiederholung möglich). Im Rahmen der 
Erklärung am Ende der Bachelor-/Master-Arbeit verpflichtet sich die
Studentin bzw.\ der 
Student, dieser, im Rahmen wissenschaftlicher Arbeit fundamentalen Pflicht,
alle Quellen angegeben zu haben und Zitate kenntlich gemacht zu haben,
nachgekommen zu sein. \\
Überwiegt bei einer Arbeit der Anteil korrekt
zitierter, aber mehr oder weniger wörtlich übernommener Passagen, ist sie zwar
im engeren Sinne kein Plagiat, allerdings auch kein Nachweis einer gemäß
Prüfungsordnung zu 
erbringenden selbstständigen wissenschaftlichen und fachpraktischen Leistung
(nicht bestanden, Wiederholung möglich). 