% Das gesetzt von Conway [Seite 39 EWolff2016:Microservices]
\chapter[Grundlagen]{Allgemeine Grundlagen zur Verwendung von Service-orientierten Systemen}
\label{chap:grundlagen}
Software zu entwickeln ist nicht immer einfach. Umso größer diese ist, umso mehr Probleme können auftreten. Es bedarf einer genauen Planung und Verständnis von Infrastruktur um eine Software mit allen Anforderungen, zufriedenstellend zu implementieren. 

Vor allem wenn es um die Weiterentwicklung und Wartung von Software geht, können große Probleme auftreten. Wurde die Architektur nicht gut gewählt oder schlecht umgesetzt, kann es das weitere Vorgehen stark beeinträchtigen, bis hin zum unmöglich machen. Es wurden daher Software-Architekturen entwickelt, welche flexibel und einfacher zu ändern sind. Außerdem kann in diesen Architekturen neue Funktionen deutlich schneller hinzugefügt werden.

\section{Die Service-orientierte Architektur}
\label{sec:architektur}
Service-orientierte Architekturen können, wenn sie richtig angewendet werden, sehr flexibel und schnell änderbar sein. Diese Architekturen zielen, wie der Name schon sagt, auf eigenständige Dienste ab, welche durch verschiedene Kommunikationskanäle miteinander kommunizieren können und dadurch die gewünschten Geschäftsprozesse abbilden. 
\begin{quotation}
    \frqq Ein Programm soll nur eine Aufgabe erledigen, und das soll es gut machen\flqq \cite[S. 2]{EWolff2015:ContinuouosDelivery}
\end{quotation}
Anstatt eine einzige große Anwendung ein zu setzten, setzt man auf viele kleine, verteile, autarke Anwendungen, welche jeweils Schnittstellen nach außen hin anbieten damit der Service genutzt werden kann. Diese Schnittstellen können unter anderem durch REST-HTTP angeboten werden.
Durch die verteilten Anwendungen funktioniert das System auch dann noch, wenn einzelne Dienste nicht verfügbar sind, jedoch bringt es ebenfalls die typischen Probleme von Verteilten Anwendungen mit sich, welche in \secref{chap:fallstudie} noch genauer erläutert werden.

\subsection{Das Gesetzt von Conway}
\label{subsec:conway}
Wie in \cite[S. 39 ff.]{EWolff2016:Microservices} beschrieben, stammt das Gesetzt von dem amerikanischen Informatiker Melvin Conway und besagt:
\begin{center}
    \textit{Organisationen, die Systeme designen, können nur solche Designs entwerfen, welche die Kommunikationsstruktur dieser Organisationen abbilden.}
\end{center}
"Conway möchte damit ausdrücken, dass die internen Kommunikationswege wichtig bei der Planung der Architektur ist. Jedes Team innerhalb einer Organisation trägt bei der Entwicklung der Architektur bei. Wird eine Schnittstelle zwischen zwei Teams benötigt, so müssen diese Teams auch kommunizieren können. Dabei müssen Kommunikationswege nicht immer offiziell sein. Oft gibt es informelle Kommunikationsstrukturen, die ebenfalls in diesem Kontext betrachtet werden können." \cite[vg. S. 39]{EWolff2016:Microservices}

\subsection{Domain-Driven Design und Bounded Context}
\label{sec:boundedContext}
Arbeitet man mit Service-Orientierten Architekturen, versucht man Services, welche zu einem bestimmten Kontext gehören, möglichst nahe beieinander zu halten. Man spricht hierbei von \textit{Bounded Context}. 
\begin{quotation}
    \frqq Bounded Context ist ein zentrales Muster in Domain-Driven Design.[..] DDD arbeitet mit großen Modellen, indem es diese in kleine verschiedene zusammengehörige Kontexte unterteilt und auf ihre Wechselwirkung unterteilt.\flqq \cite{mfowler:BoundedContext}
\end{quotation}

\begin{figure}[htb]
    \centering 
    \includegraphics[width=\linewidth]{content/images/BoundedContext}\
    \quelle\url{http://martinfowler.com/bliki/BoundedContext.html}
    \caption[Bounded Context]{Bounded Context\\}
    \label{fig:BoundedContext}  
\end{figure} 
In dieser Grafik wird noch einmal der Begriff Bounded Context genauer verdeutlicht. Es existieren zwei eigenständige Prozesse. Auf der linken Seite der Sales Kontext und auf der rechten Seite der Support Kontext. Jeder Kontext besitzt verschiedene Services, welche benötigt werden um den Prozess durchführen zu können. Lediglich zwischen den \textit{Customer} und \textit{Product} Services besteht eine Verbindung der beiden Prozesse.