\chapter{Grundlagen}
\label{chap:grundlagen}
Software zu entwickeln ist nicht immer einfach. Umso größer diese ist, umso mehr Probleme können auftreten. Es bedarf einer genauen Planung und Verständnis der Infrastruktur um eine Software mit allen Anforderungen, zufriedenstellend zu implementieren. Das heißt aber nicht, dass kleine Programme keine Probleme bereiten können. 

Bevor ein Programm entwickelt werden kann, müssen verschiedene Schritte durchlaufen werden. Es muss zunächst ein bedarf für die Software bestehen. Sei es Unternehmens-Software, welche eingesetzt wird um eine bestimmte Aufgabe zu erledigen oder aber um eine neue Technologie zu testen, ohne das diese Produktiv eingesetzt wird. Im jeden Fall benötigt man ein Kontext für das zu entwickelnde Programm und damit auch die Anforderungen an dieses. Nach dem erfassen dieser, muss die interne Architektur der Software geplant und schließlich umgesetzt werden.

\section{Herausforderung}
Je nach Komplexität der Anforderungen an eine Software, ist die interne Aufteilung des Programms wichtig, um die Wartbarkeit zu gewährleisten und Redundanzen innerhalb der Software zu vermeiden.
Aber auch einfache Anforderungen können zur Herausforderung werden, wenn diese vorsehen, das eine Software für einen weit größeren Kontext ausgelegt sein soll, als sie genutzt wird. Dies kann dazu führen, dass bei steigender Nutzung Probleme auftreten, welche nicht vorhersehbar waren, weil sie bis dahin noch nicht im Unternehmen aufgetreten sind. Das führt häufig dazu, dass die Software durch Aufwendung von Zeit und Geld, erneut entwickelt werden muss. 

\section{Architektur}
