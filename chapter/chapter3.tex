% Das gesetzt von Conway [Seite 39 EWolff2016]
\chapter{Grundlagen}
\label{chap:grundlagen}
Software zu entwickeln ist nicht immer einfach. Umso größer diese ist, umso mehr Probleme können auftreten. Es bedarf einer genauen Planung und Verständnis von Infrastruktur um eine Software mit allen Anforderungen zufriedenstellend zu implementieren. 

Vor allem wenn es um die Weiterentwicklung und Wartung von Software geht, können große Probleme auftreten. Wurde die Architektur nicht gut gewählt oder schlecht umgesetzt, kann es das weitere Vorgehen stark beeinträchtigen, bis hin zum unmöglich machen. Es wurden daher Software-Architekturen entwickelt, welche flexibel und einfacher zu ändern sind. Außerdem kann in diesen Architekturen neue Funktionen deutlich schneller hinzugefügt werden.

\section{Architektur}
\label{sec:architektur}
Die rede ist von Microservice-Architekturen und \SOA (SOA). Beide Architekturen zielen, wie der Name schon sagt, auf eigenständige Services ab, welche durch verschiedene Kommunikationskanäle miteinander kommunizieren und dadurch die gewünschten Geschäftsprozesse abbilden. "Die Grundidee dahinter ist, das ein Programm nur eine einfache Aufgabe erledigen soll, und das soll es gut machen." (vgl. \cite[S. 2]{EWolff2015})

\section{Swagger}

%Zookeeper, etcd
%Spring Cloud Config
\section{Service Discovery - Spring}

\section{Continous-Delivery}

\section{Weitere Werkzeuge}

\subsection{Docker}

\subsection{Vagrant}

\subsection{puppet}

\subsection{ELK (ElasticSearch, LogStash, Kibana)}

