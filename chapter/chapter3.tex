\chapter{Implementierung neuronaler Netze}
\label{chap:implementierung}

\section{Die Wahl der Programmiersprache: Java}
Kurze Begründung für die Auswahl der Programmiersprache. Mit welchem
Entwicklungstool wurde gearbeitet und warum? Oder wurde direkt mit dem JDK 
gearbeitet? Wenn ja, warum? {\bf Keine} detaillierte
Einführung in Java; das ist inzwischen Standard. Allerdings: Neue und
spezielle Bibliotheken, Packages oder Klassen, die benutzt werden, müssen
begründet und erläutert werden.

\section{Details zum Entwurf und zur Entwicklung}
Klassische Vorgehensweise bei dem Entwurf und der Entwicklung eines
Anwendungsprogramms (OOA, OOD, OOP, etc.). 
Insbesondere sollten hier (oder -- falls zu umfangreich -- spätestens im
Anhang) die entsprechenden Diagramme eingebunden werden.

\section{Details zur konkreten Implementierung}
Hier sollten ausgewählte Teile des Source-Codes, die für die Funktionalität
des Programms fundamental sind, im Detail erläutert werden. Neben dem
Aufzeigen der generellen Konzepte zur Umsetzung des mathematischen Kalküls in
Programm-Code geht es hier auch um Fragen wie Effizienz, Parallelisierbarkeit,
numerische Stabilität, etc.. 

\begin{lstlisting}[firstnumber=1, caption={[Hallo-Welt-Programm (Java)] Ein Hallo-Welt-Programm in der Programmiersprache Java.}, label=lst:hallo_welt, style=java]
public final class HelloWorld
{
    /*
     * Ein Umlaut Test: Ä
     */
	public static void main(final String[] arg)
	{
		System.out.println("Hallo Welt!"); // Ausgabe: Hallo Welt!
	}
}
\end{lstlisting}
%************** XML ******************
\subsection{XML}
Beispiel für XML-Code siehe Quelltext

\lstinputlisting[caption=Beispiel für XML-Code, label=lst_xml_code, style=xml]{content/listings/xml_code.xml}


%************** JAVA ******************
\subsection{JAVA}
Beispiel für Java-Code siehe Quelltext

\lstinputlisting[caption=Beispiel für Java-Code, label=lst_java_code, style=java]{content/listings/java_code.java}


\section{Probleme bei der Implementierung}
Klar!

\section{Zeichnungen}

Die folgenden Zeichnungen wurden mit den \LaTeX -Zusatzpaketen pgf und tikz erstellt. Sie stellen sehr mächtige Werkzeuge zur Verfügung um Diagramme und Grafiken aller Art zu erstellen. Die Ergebnisse sind professionell und können, falls nötig, mit wenig Aufwand geändert werden. Es erfordert natürlich eine gewisse Einarbeitung, aber diese wird durch die Resultate schnell wieder aufgewogen.
Eine umfangreiche Anleitung mit vielen weiteren Beispielen findet sich auf \\ \href{http://www.ctan.org/tex-archive/graphics/pgf/base/doc/generic/pgf/pgfmanual.pdf}{http://www.ctan.org/tex-archive/graphics/pgf/base/doc/generic/pgf/pgfmanual.pdf} \\

Es folgen einige Beispiele.


%********************** Zustandsdiagramm *******************
\subsection{Zustandsdiagramm}

Das Zustandsdiagramm (englisch: state diagram) der UML ist eine der dreizehn Diagrammarten dieser Modellierungssprache für Software und andere Systeme. Es stellt einen endlichen Automaten in einer UML-Sonderform grafisch dar und wird benutzt, um entweder das Verhalten eines Systems oder die zulässige Nutzung der Schnittstelle eines Systems zu spezifizieren.

\begin{figure}[H]
    \begin{center}
        \begin{tikzpicture}[shorten >=1pt,node distance=2cm,on grid,>=stealth', every state/.style={draw=blue!50,very thick,fill=blue!20}]
        \node[state,initial] (q_0) {$q_0$};
        \node[state] (q_1) [above right=of q_0] {$q_1$};
        \node[state] (q_2) [below right=of q_0] {$q_2$};
        \path[->] (q_0) edge node [above left] {0} (q_1)
        edge node [below left] {1} (q_2)
        (q_1) edge [loop above] node {0} ()
        (q_2) edge [loop below] node {1} ();
        \end{tikzpicture}
        \caption{Zustandsdiagramm}
        \label{fig:zustandsdiagramm}
        \end{center}
        \end{figure}
        
        
        %********************** Petrinetz *******************
        \subsection{Petrinetz}
        Ein Petri-Netz ist ein mathematisches Modell von nebenläufigen Systemen. Es ist eine formale Methode der Modellierung von Systemen bzw. Transformationsprozessen. Die ursprüngliche Form der Petri-Netze nennt man auch Bedingungs- oder Ereignisnetz. Petri-Netze wurden durch Carl Adam Petri in den 1960er Jahren definiert. Sie verallgemeinern wegen der Fähigkeit, nebenläufige Ereignisse  darzustellen, die Automatentheorie.
        
        \begin{figure}[H]
            \begin{center}
                \begin{tikzpicture}
                [node distance=1.3cm,on grid,>=stealth',bend angle=45,auto,
                every place/.style= {minimum size=6mm,thick,draw=blue!75,fill=blue!20},
                every transition/.style={thick,draw=black!75,fill=black!20},
                red place/.style= {place,draw=red!75,fill=red!20},
                every label/.style= {red}]
                \node [place,tokens=1] (w1) {};
                \node [place] (c1) [below=of w1] {};
                \node [place] (s) [below=of c1,label=above:$s\le 3$] {};
                \node [place] (c2) [below=of s] {};
                \node [place,tokens=1] (w2) [below=of c2] {};
                \node [transition] (e1) [left=of c1] {}
                edge [pre,bend left] (w1)
                edge [post,bend right] (s)
                edge [post] (c1);
                \node [transition] (e2) [left=of c2] {}
                edge [pre,bend right] (w2)
                edge [post,bend left] (s)
                edge [post] (c2);
                \node [transition] (l1) [right=of c1] {}
                edge [pre] (c1)
                edge [pre,bend left] (s)
                edge [post,bend right] node[swap] {2} (w1);
                \node [transition] (l2) [right=of c2] {}
                edge [pre] (c2)
                edge [pre,bend right] (s)
                edge [post,bend left] node {2} (w2);
                \begin{scope}[xshift=6cm]
                \node [place,tokens=1] (w1') {};
                \node [place] (c1') [below=of w1'] {};
                \node [red place] (s1') [below=of c1',xshift=-5mm]
                [label=left:$s$] {};
                \node [red place,tokens=3] (s2') [below=of c1',xshift=5mm]
                [label=right:$\bar s$] {};
                \node [place] (c2') [below=of s1',xshift=5mm] {};
                \node [place,tokens=1] (w2') [below=of c2'] {};
                \node [transition] (e1') [left=of c1'] {}
                edge [pre,bend left] (w1')
                edge [post] (s1')
                edge [pre] (s2')
                edge [post] (c1');
                \node [transition] (e2') [left=of c2'] {}
                edge [pre,bend right] (w2')
                edge [post] (s1')
                edge [pre] (s2')
                edge [post] (c2');
                \node [transition] (l1') [right=of c1'] {}
                edge [pre] (c1')
                edge [pre] (s1')
                edge [post] (s2')
                edge [post,bend right] node[swap] {2} (w1');
                \node [transition] (l2') [right=of c2'] {}
                edge [pre] (c2')
                edge [pre] (s1')
                edge [post] (s2')
                edge [post,bend left] node {2} (w2');
                \end{scope}
                \begin{pgfonlayer}{background}
                \node (r1) [fill=black!10,rounded corners,fit=(w1)(w2)(e1)(e2)(l1)(l2)] {};
                \node (r2) [fill=black!10,rounded corners,fit=(w1')(w2')(e1')(e2')(l1')(l2')] {};
                \end{pgfonlayer}
                \draw [shorten >=1mm,-to,thick,decorate,
                decoration={snake,amplitude=.4mm,segment length=2mm,
                    pre=moveto,pre length=1mm,post length=2mm}]
                    (r1) -- (r2) node [above=1mm,midway,text width=3cm,text centered]
                    {replacement of the \textcolor{red}{capacity} by \textcolor{red}{two places}};
                    \end{tikzpicture}
                    \caption{Petrinetz}
                    \label{fig:petrinetz}
                    \end{center}
                    \end{figure}
                    
                    
                    
                    %********************** Graph *******************
                    \subsection{Graph}
                    
                    Ein Graph besteht in der Graphentheorie anschaulich aus einer Menge von Punkten, zwischen denen Linien verlaufen. Die Punkte nennt man Knoten  oder Ecken, die Linien nennt man meist Kanten, manchmal auch Bögen. Auf die Form der Knoten und Kanten kommt es im allgemeinen dabei nicht an. Knoten und Kanten können auch mit Namen versehen sein, dann spricht man von einem benannten Graphen.
                    
                    \begin{figure}[H]
                        \begin{center}
                            \begin{tikzpicture}[domain=0:4,label/.style={postaction={
                                    decorate,
                                    decoration={
                                        markings,
                                        mark=at position .75 with \node #1;}}}]
                                        \draw[very thin,color=gray] (-0.1,-1.1) grid (3.9,3.9);
                                        \draw[->] (-0.2,0) -- (4.2,0) node[right] {$x$};
                                        \draw[->] (0,-1.2) -- (0,4.2) node[above] {$f(x)$};
                                        \draw[red,label={[above left]{$f(x)=x$}}] plot (\x,\x);
                                        \draw[blue,label={[below left]{$f(x)=\sin x$}}] plot (\x,{sin(\x r)});
                                        \draw[orange,label={[right]{$f(x)= \frac{1}{20} \mathrm e^x$}}] plot (\x,{0.05*exp(\x)});
                                        \end{tikzpicture}
                                        \caption{Graph}
                                        \label{fig:graph}
                                        \end{center}
                                        \end{figure}
