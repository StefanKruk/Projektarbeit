\chapter{Ergebniss}
\label{chap:Ergebniss}
Nachdem die Technologien Microservice und SOA nun ausführlich beschrieben und verglichen wurden, muss überprüft werden, ob die Fragen aus Kapitel \secref{sec:dasProblem} beantwortet werden können. Dabei wurde die Frage auf die Unterschiede der beiden Paradigmen bereits in Kapitel \secref{chap:vergleich} beantwortet. Es müssen jedoch noch folgende Fragestellungen beantwortet werden:

\begin{enumerate}
    \item Welche Vor- und Nachteile hat das jeweilige Paradigma?
    \item Gibt es Grenzen oder Beschränkungen bei der Benutzung der genannten Paradigmen?
    \item In wie weit helfen die Paradigmen die angesprochenen Problematiken zu lösen?
\end{enumerate}


\section{Welche Vor- und Nachteile hat das jeweilige Modell?}
\label{sec:VorNachteile}
% Business aufstellung umstellen auf das jeweilige Modell
Beide Paradigmen sind, wie bereits geschrieben "`Service-orientierte Modelle"'. Damit ist gemeint, dass eine Anwendung in mehrere Dienste aufgeteilt ist. In klassischen monolithischen Anwendungen kann es passieren, dass ein und dieselbe Funktion in verschiedenen Unternehmensanwendungen existiert.
\\\\
Der Vorteil bei Microservices und SOA ist, dass eine Funktion in der gesamten Infrastruktur nur einmal existiert. Dadurch ist es möglich bereits vorhandene Funktionen auf einfache Weise zu ändern ohne eine Vielzahl von Anwendungen, die diese Funktion benötigen, zu ändern und neu zu deployen.
Ein weiterer Vorteil liegt in der Skalierung. Dadurch, dass Funktionen in einzelnen Diensten verpackt werden, ist es möglich, viel genutzte Funktionen und damit viel genutzte Dienste zu identifizieren und zu skalieren. In monolithischen Systemen muss immer die gesamte Anwendung skaliert werden, selbst wenn nur einzelne Teile beansprucht werden.
Dadurch, dass für eine Funktion ein Dienst existiert, müssen diese nicht in der gleichen Programmiersprache geschrieben worden sein. Es können zum Beispiel viele unterschiedliche Sprachen verwendet werden. Die Dienste benötigen dabei standardisierte Schnittstellen, wie zum Beispiel REST-HTTP oder SOAP. Durch standardisierte Schnittstellen können jegliche Dienste diese nutzen ohne dass eine zusätzliche Brücke dafür gebaut werden muss.
\\\\
Der Nachteil von Service-orientierten Systemen liegt in den Kommunikationswegen. Fällt zum Beispiel ein Dienst in diesen Systemen aus oder wird die Kommunikation zu diesen unterbrochen, kann es zu Teilausfällen im gesamten System kommen. Ist ein kritischer Dienst betroffen und es existieren keine weiteren Instanzen dieses Dienstes, kommt es im schlimmsten Fall zu einem Ausfall des gesamten Systems. Die Erkennung von Ausfällen ist ein weiteres Problem in solchen Systemen.
\\\\
Bei SOA existieren zudem oft Anwendungen wie ERP- oder COBOL-Systeme. Auch diese müssen dem SOA-System hinzugefügt werden. Damit dies funktioniert, müssen oft Wrapper geschrieben werden, welche "`um diese Systeme gelegt"' werden und die Kommunikation, sowie die Interpretation der Nachrichten steuern. Zusätzlich werden diese Wrapper dazu eingesetzt fehlerhafte Nachrichten bzw. Anweisungen heraus zu filtern.

\section{Gibt es Grenzen oder Beschränkungen bei der Benutzung der genannten Paradigmen?}
\label{sec:Beschraenkungen}
Einschränkungen gibt es in den Paradigmen hauptsächlich durch die Kommunikation zwischen den einzelnen Diensten. Wie bereits in \secref{sec:FazitKommunikation} erläutert können Microservice-Dienste ohne Einschränkungen miteinander kommunizieren und Daten/Nachrichten austauschen. Bei SOA hingegen existiert ein Enterprise Service Bus (ESB). Alle Nachrichten werden durch den ESB an die jeweiligen Dienste verteilt. Eine direkte Kommunikation unter den Diensten ist nicht gewollt.
\\\\
Im Falle von SOA spielt nicht nur die IT eine Rolle. Da SOA ursprünglich aus dem Businessbereich stammt müssen die Geschäftsprozesse so umgestellt werden, dass diese in das neue Modell passen. Zudem muss sich die IT dem Business an nähren und ihre Systeme auf das neue Modell umstellen. Dies sollte gut überlegt werden, da hiermit häufig ein großer und schwieriger Prozess verbunden ist. Oft ist es nicht einfach alle Geschäftsprozesse und die IT-Infrastruktur auf dieses Modell umzustellen.

\section{In wie weit helfen die Paradigmen die angesprochenen Problematiken zu lösen?}
\label{sec:LoesungDerProblematiken}
In Kapitel \secref{sec:motivation}\ wurde die Problematik der Time-to-Market Zeitspanne genannt. Paradigmen wie Microservice und SOA können hier durchaus diese Problematik lösen. Die Time-to-Market Zeitspanne ist die Zeit, in der die Wertschöpfungskette durchlaufen wird. Je schneller dies geschieht, je geringer ist die Zeitspanne.
\\\\
Mit Hilfe von SOA sollen alle Unternehmenskomponenten möglichst optimal miteinander kommunizieren. Damit dies geschieht, müssen die einzelnen Unternehmensprozesse standardisiert werden. Dadurch kann ebenfalls die Wertschöpfungskette standardisiert durchlaufen und im laufe dessen, die Prozesse optimiert werden. Dadurch kann die Wertschöpfungskette in möglichst optimaler Zeit durchlaufen werden.
\\\\
Bei einem Microservice-System hingegen, wird eine Anwendung in verschiedene Module (Services) aufgeteilt. Dadurch können diese schnell gewartet werden, ohne das gesamte System offline zu nehmen oder neu deployen zu müssen. Außerdem können neue Funktionen schnell hinzugefügt werden, ohne die Integrität der anderen Dienste zu gefährden. Hierbei beschreibt die Time-to-Market Zeitspanne, die Zeit bis eine neue Funktion implementiert und hinzugefügt wurde. Bei unserem fiktiven Unternehmen, der \textit{Auktionen GmbH}\ könnte dies zum Beispiel die Möglichkeit sein, regional Beschränkte Angebote zu erstellen.
\\\\
Damit die Time-to-Market (TTM) Zeitspanne möglichst gering gehalten wird und ein Benutzer nicht durch hinzufügen vieler neuer Funktionen zur gleichen Zeit überfordert ist, wird oft ein Continuous Delivery/Deployment Ansatz verwendet. Bei diesem Ansatz wird ein Feature Deployed und zur Anwendung hinzugefügt, sobald dieses fertig ist. Einzelne Funktionen sind oft schnell erstellt, jedoch werden diese in monolithischen Systemen oft nur in regelmäßigen Abständen, wie zum Beispiel einmal im Monat deployed. Dadurch beträgt die TTM Zeitspanne immer ein Monat und sollte ein Feature nicht rechtzeitig fertig werden, verlängert sich diese um einen weiteren Monat. Da in Service-orientierten Systemen Features deployed werden können, sobald sie fertig sind, beträgt die maximale TTM Zeitspanne genau solange, wie die Entwicklungsdauer eines Features. Es sollte jedoch auf die Versionierung der Schnittstellen geachtet werden, denn wenn diese geändert werden, kann es dazu führen, dass andere Dienste nicht mehr darauf zugreifen können. Oft werden daher ältere Schnittstellen weiterhin gepflegt und zu einem späteren Zeitpunkt, sobald kein Dienst mehr auf diese zugreift, entfernt.

\section{Fazit}
\label{sec:Fazit}
Beide Architekturen unterscheiden sich hinsichtlich ihrer Nutzung und Handhabung. So ist das Microservice Modell ein Entwicklungs-Modell. Es unterstützt die IT-Abteilung bei der Entwicklung von Software, indem eine Anwendung in verschiedene Dienste aufgeteilt wird. Dies ähnelt den herkömmlichen monolithischen Anwendungen insofern, dass in diesen einzelne Funktionen in Pakete oder Bibliotheken verpackt sind. In der Microservice-Architektur sind diese in Dienste (Microservices) verpackt. Dagegen ist SOA ein Business-Modell. Es kommt aus dem Businessbereich und soll damit auch diesem Helfen. Hierbei sind Dienste keine "`Microservices"' sondern zum Teil große ERP- oder COBALT-Systeme.
\\\\
Das Einsatzgebiet der jeweiligen Paradigmen ist sehr verschieden, wodurch ein direkter Vergleich nicht möglich ist. Man kann jedoch kann aus den gesammelten Erkenntnissen schließen, dass beide Paradigmen Schnittpunkte haben. Diese Schnittpunkte sind zum Beispiel der Aufbau auf einzelnen Diensten. Dies zeigt jedoch auch, dass, obwohl Schnittpunkte existieren, diese sehr unterschiedlich sind.
\\\\
Microservices sind eine gute Möglichkeit, um Software zu entwickeln, welche sich oft schnell weiterentwickeln und verändern muss, sowie Software die große Nutzerlasten auf einzelne zentrale Teile aushalten muss. In diesen Anwendungsfällen ist es oft problematisch monolithische Systeme einzusetzen, da sie mit steigender Größe weniger dynamisch werden.
SOA hingegen ist ein System welches vor allem im Businessbereich anklang findet und nicht als IT Architekturmodell gesehen werden sollte sondern als Herangehensweise. Dies liegt an der Tatsache, dass die vorhandenen Geschäftsprozesse einen großen Teil dieses Systems ausmachen.
\\\\
Bei beiden Paradigmen sollte jedoch darauf geachtet werden, dass die Organisation und der Aufbau des Basissystems (Service Discovery, Ausfallsicherheit, etc.) nicht dem eigentlichen Nutzen entgegen wirkt oder deutlich übersteigt. Keiner der beiden Paradigmen sollte eingesetzt werden, wenn kein Bedarf vorhanden ist. Zudem muss beachtet werden, dass mit dem Einsatz dieser Paradigmen neue Probleme entstehen, welche gelöst werden müssen.
\\\\
Weder Microservices, noch SOA ist eine Allround-Lösung für jegliche Probleme. Es können zwar gewisse Problemfelder, wie sie schon erläutert wurden, gelöst werden, jedoch nicht für alle, welche in der IT oder im Business vorhanden sind. Es sollte daher vor Einführung eines der Paradigmen eine Machbarkeitsstudie durchgeführt.
\\\\
Grundsätzlich schließen sich beide Systeme nicht aus. Mit SOA sollen möglichst alle Anwendungen eines Unternehmens mit einander verbunden werden, während ein Microservice-System eine Anwendung in viele Teilanwendungen, auch Dienste genannt, aufteilt. Dementsprechend kann ein Microservice-System ein Teil von einem SOA-System sein. Gerade wenn die Frage aufkommt, wie das neu entwickelte System in die bereits bestehende Systemlandschaft eingebunden wird, kann SOA ein guter Ansatz sein um das System einzubinden.