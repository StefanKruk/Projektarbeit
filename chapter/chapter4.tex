\chapter{Microservices}
\label{chap:Microservices}

%% PUSH-Architektur (Event Bus) / PULL-Architektur (Rest-Services)

% [Seite 66 EWolff2016] Technologische Wahlfreiheit
\subsection{Herausforderung}
Wie in \cite[S. 25]{EWolff2016:Microservices} beschrieben, besteht ein wesentliches Problem in der Kommunikation zwischen den Services. Der Ausfall eines Services kann im schlechtesten Fall dazu führen, dass alle anderen Microservices nicht mehr funktionieren. Um das zu verhindern, muss klar definiert werden


\section{Continuous-Delivery}
\label{sec:ContinuousDelivery}
"Wer bisher nur Deployment-Monolithen betrieben hat, ist bei Microservices damit konfrontiert, dass es sehr viel mehr deploybare Artefakte gibt, weil jeder Microservice unabhängig in Produktion gebracht wird"\cite[S. 241]{EWolff2016:Microservices}
"Unabhängiges Deployment ist ein zentrales Ziel von Microservices. Außerdem muss das Deployment automatisiert sein, weil ein manuelles Deployment oder auch nur manuelle Nacharbeitung aufgrund der großen Anzahl Microservices nicht umgesetzt werden können"\cite[S. 256]{EWolff2016:Microservices}

\section{Orchestration vs Choreographie}
\label{sec:orchestrationvschoreographie}

