\chapter{Anwendung neuronaler Netze}
\label{chap:anwendung}

\section{Anforderungen an Hard- und Software}
\begin{itemize}
\itemsep -6pt
\item Hardware-Plattform
\item Besonderheiten der Rechner-Architektur
\item Prozessor-Typ bzw.\ -Typen 
\item Taktfrequenz 
\item Hauptspeicher-Größe 
\item Cache 
\item Swapspace (falls verwendet)
\item Betriebssystem und Versionsnummer 
\item Entwicklungsumgebung 
\item Programmiersprache 
\item Compiler bzw.\ Interpreter und Versionsnummer 
\item Compiler- bzw.\ Interpreteroptionen 
\item verwendete Bibliotheken
\item alle relevanten Implementationsdetails 
      (insbesondere gesetzte Parameter u.~ä.)
\item usw., usw.
\end{itemize}
Alle diese Angaben sind wichtig, da sich die Ergebnisse sonst nur bedingt
reproduzieren lassen. Nicht reproduzierbare Anwendungen sind aber wertlos!

\section{Anwendung des Java-Programms}

Hier sollte eine Beispiel-Anwendung mit dem entwickelten Programm
durchgespielt werden und möglichst mit Screenshots dokumentiert werden.
