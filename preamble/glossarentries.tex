%%%%%%%%%%%%%%%%%%%%%%%%%%%%%%%%%%%%%%%%%%%%%%%%%%%%%%%%%%%%%%%%%%%%%%%%%%%%%%%
% ****************************** Glossarie ************************************
%%%%%%%%%%%%%%%%%%%%%%%%%%%%%%%%%%%%%%%%%%%%%%%%%%%%%%%%%%%%%%%%%%%%%%%%%%%%%%%

%Befehle für Glossar
\newglossaryentry{glos:SOA}{
    name=SOA,
    description={\textit{SOA} steht für Service-orientierte Architektur und ist ein Architekturmuster für Service-orientierte Systeme. SOA ist in dem Bereich der verteilten Systeme einzuordnen und wird eingesetzt, um Dienste von IT-Systemen zu strukturieren und zu nutzen.}
}

\newglossaryentry{glos:Microservice}{
    name=Microservice, 
    description={\textit{Microservices} sind ein Architekturmuster aus dem Bereich der verteilten Anwendungen, bei dem komplexe Anwendungen in eigenständige Prozessen zerlegt werden. Diese kommunizieren mit Hilfe von sprachunabhängigen Schnittstellen untereinander. Die Prozesse, welche ebenfalls Microservices genannt werden, sind kleine unabhängige Dienste, welche möglichst nur eine Aufgabe erledigen sollen, diese aber dafür besonders gut. Dadurch ist ein modularer Aufbau von Anwendungssoftware möglich.}
}

\newglossaryentry{glos:Services}{
	name=Services,
	description={\textit{Services}, zu deutsch Dienste, sind IT-Repräsentanten von fachlichen Funktionalitäten. Dabei sollen Services in sich abgeschlossen (autark) sein, damit diese eigenständig genutzt werden können. Um Services über das Netzwerk nutzen zu können, müssen wohldefinierte Schnittstellen vorhanden sein. Zusätzlich sollte die Granularität von Services grob sein, um die Abhängigkeiten zu anderen Diensten gering zu halten.}
}

\newglossaryentry{glos:APIExposureGateway}{
	name={API Exposure Gateway},
	description={Ein \textit{API Exposure Gateway}\ ist eine Schicht in einem System, welches eine zentrale Stelle für die Veröffentlichung von APIs darstellt. Damit muss der Nutzer nicht die dahinter liegenden Server kennen, um deren APIs aufrufen zu können.}
}