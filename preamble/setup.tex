% **************************** HYPERREF SETUP *******************************
\definecolor{linkcolor}{rgb}{1,0.5,0}
\hypersetup
{
bookmarks=true,                        % Lesezeichen im PDF erzeugen
bookmarksopen=true,                    % Lesezeichen im PDF sofort anzeigen
backref=true,                          % Rückverweise im Literaturverzeichnis
colorlinks=true,                       % Farbige Verweise
%hidelinks = true,                      % Verweise verbergen (entfernt Farbe und Rahmen)
pdfstartview={FitH},                   % Ansicht des PDFs beim öffnen
pdftitle={\titel},                     % Title des PDFs
pdfauthor={\author , \supervisor},     % Autor des PDFs
pdfsubject={\subject},                 % Thema des PDFs
%pdfcreator={Creator},                 % Erzeuger des Dokuments (Anwendungsprogramm)
%pdfproducer={Producer},               % Ersteller des PDFs (Programm/Bibliothek/Skript)
pdfkeywords={\keywords},               % Stichwörter zum PDF
linkcolor=linkcolor,                   % Farbe von Querverweisen
citecolor=linkcolor,                       % Farbe von Zitaten
filecolor=magenta,                     % Farbe von Verweisen auf Dateien
urlcolor=cyan                          % Farbe von URLs
}
% Weitere Optionen: http://www.tug.org/applications/hyperref/manual.html

% Für Zeichnungen
\usetikzlibrary{% 
    arrows,% 
    calc,% 
    fit,% 
    patterns,% 
    plotmarks,% 
    shapes.geometric,% 
    shapes.misc,% 
    shapes.symbols,% 
    shapes.arrows,% 
    shapes.callouts,% 
    shapes.multipart,% 
    shapes.gates.logic.US,% 
    shapes.gates.logic.IEC,% 
    er,% 
    automata,% 
    backgrounds,% 
    chains,% 
    topaths,% 
    trees,% 
    petri,% 
    mindmap,% 
    matrix,% 
    calendar,% 
    folding,% 
    fadings,% 
    through,% 
    positioning,% 
    scopes,% 
    decorations.fractals,% 
    decorations.shapes,% 
    decorations.text,% 
    decorations.pathmorphing,% 
    decorations.pathreplacing,% 
    decorations.footprints,% 
    decorations.markings,% 
    shadows
} 

% **************************** LISTINGS SETUP *******************************
\definecolor{keywords}{rgb}{0.5 0 0.3}
\definecolor{comments}{rgb}{0.25,0.5,0.37}
\definecolor{lila}{RGB}{112, 6, 147}
\definecolor{kommentgreen}{RGB}{5,132,71}
\definecolor{grey}{RGB}{242,242,242}  
\definecolor{darkgreen}{named}{green}
\definecolor{darkblue}{named}{blue}
\definecolor{lightblue}{RGB} {63,95,191}
\definecolor{darkred}{named}{red}
\definecolor{grau}{named}{gray}
\definecolor{fh_orange}{rgb}{0.953,0.201,0}
\definecolor{fh_grau}{rgb}{0.76,0.75,0.76}

\definecolor{listinggray}{gray}{0.9}
\definecolor{lbcolor}{rgb}{0.9,0.9,0.9}
\lstset{literate=%
    {Ö}{{\"O}}1
    {Ä}{{\"A}}1
    {Ü}{{\"U}}1
    {ß}{{\ss}}1
    {ü}{{\"u}}1
    {ä}{{\"a}}1
    {ö}{{\"o}}1
    {~}{{\textasciitilde}}1
}
\lstset{ %
    backgroundcolor=\color{grey},   % Hintergrundfarbe
    basicstyle=\linespread{0.94}\footnotesize\ttfamily, % Schrifteinstellungen für Quellcode
    breakatwhitespace=false,         % Automatische Zeilenumbrüche nur bei Leer- oder Tabulatorzeichen (Leerraum/whitespaces)
    breaklines=true,                 % Automatische Zeilenumbrüche
    captionpos=b,                    % Beschriftung unten
    commentstyle=\color{comments},   % Schrifteinstellungen für Kommentare
    columns=felxible,                 % Ist notwendig, damit man Quellcode aus den Listings kopieren kann
    %  deletekeywords={...},            % Bestimmte Schlüsselwörter entfernen
    escapeinside={\%*}{*)},          % Defintion von Escape-Sequenzen
    extendedchars=false,                   % Nicht ASCII-Zeichen erlauben
    frame=single,                    % Rahmen um den Quellcode
    keepspaces=true,                 % Einrückungen im Quellcode behalten
    keywordstyle=\bfseries\color{keywords},% Schrifteinstellungen für Schlüsselwörter
    language=java,                   % Programmiersprache des Quellcodes
    %  morekeywords={*,...},            % Zusätzliche Schlüsselwörter
    numbers=left,                    % Zeilennummerierung
    numbersep=5pt,                   % Abstand zwischen Zeilennummerierung und Quellcode
    numberstyle=\color{black}, % Schrifteinstellungen für Zeilennummern
    rulecolor=\color{black},         % if not set, the frame-color may be changed on line-breaks within not-black text (e.g. comments (green here))
    showspaces=false,                % Leerraum-Zeichen anzeigen
    showstringspaces=false,          % Leerzeichen in Zeichenketten anzeigen
    showtabs=false,                  % Tabulatorzeichen in Zeichenketten anzeigen
    stepnumber=1,                    % Schrittweite bei Zeilennummern
    stringstyle=\color{blue},        % Schrifteinstellungen für Zeichenketten
    tabsize=4,                       % Tabulatorbreite (Anzahl Leerzeichen)
    numberbychapter=false            % Nummeriere Quellcode fortlaufend je Kapitel
}

\lstdefinestyle{java}
{
    language=Java,
    keywordstyle=\bfseries\color{lila},  	% underlined bold black keywords 
    identifierstyle=\bfseries\color{blue}, 
    commentstyle=\bfseries\color{kommentgreen}, % white comments 
    stringstyle=\bfseries\color{black},
}

\lstdefinestyle{xml}
{
    language=xml,
    basicstyle=\fontsize{9pt}{9pt}\selectfont\color{kommentgreen},
    keywordstyle=\color{lila},  	% underlined bold black keywords 
    %Hier können bei Bedarf noch weitere Keywords eingetragen werden
    keywords={name, value, version, encoding, id, type, xmlns:xsi, ref, namespace},
    identifierstyle=\color{black},  
    stringstyle=\color{blue},  
    commentstyle=\color{lightblue},
    morecomment=[s]{<!--}{-->},
    rulecolor=\color{black}
}
\renewcommand{\lstlistlistingname}{Quellcode}
\renewcommand{\lstlistingname}{Quellcode}

\AtBeginDocument{\numberwithin{lstlisting}{section}} % Nummeriere Quellcode fortlaufend je Abschnitt

% ************************** HEADER/FOOTER SETUP ****************************
\setlength{\headheight}{15pt}

\renewcommand{\chaptermark}[1]{ \markboth{#1}{} }

\fancyhf{}
\fancyhead[LE]{\thepage \ \ \ \ {\tiny \author, \today}}
\fancyhead[RO]{{\tiny \author, \today} \ \ \ \ \thepage}
\fancyhead[LO,RE]{\textit{\nouppercase{\leftmark}} }
\renewcommand{\headrulewidth}{0pt}

% **************************** GRAPHICX SETUP *********************************
\DeclareGraphicsExtensions{.pdf,.png,.jpg} % bekannte Graphik-Dateiformate (müssen nicht mehr im Dateinamen angegeben werden, also statt "beispiel.png" nur noch "beispiel")
\graphicspath{{./figure/}}   % path to graphics folder, usage {PATH},{ANOTHERPATH}...

% ************************** BIBLIOGRAPHY SETUP ********************************
\bibliographystyle{dinat}    % Literaturverzeichnis nach DIN
%\AtBeginDocument{\nocite{*}}    % Diese Zeile vor der Abgabe der Arbeit entfernen!

% ****************************** MATH SETUP ************************************
\everymath{\displaystyle}    % Erzwinge \displaystyle für Mathematischen-Modus


% ************************* THEOREMS AND PROOF *********************************
\newtheoremstyle{thesis}     % Name des neuen Theorem-Stils
{3pt}                        % Abstand oberhalb des Theorems
{3pt}                        % Abstand unterhalb des Theorems
{\itshape}                   % Schrifteinstellungen innerhalb des Theorems
{}                           % Einrückung der Theorem-Überschrift
{\bfseries}                  % Schrifteinstellungen für die Überschrift des Theorems
{}                           % Satzzeichen zwischen Überschrift und Theorem-Rumpf
{\newline}                   % Abstand hinter der Überschrift
{}                           % Spezifikation der Überschrift
  
\theoremstyle{thesis}        % Verwende neuen Theorem-Stil

\newtheorem{theorem}{Satz}[section] % neue Theorem-Umgebung: theorem (Satz)
\providecommand*{\theoremautorefname}{Satz} % autoref-Name für theorem

\newtheorem{definition}{Definition}[section] % neue Theorem-Umgebung: definition (Definition)
\providecommand*{\definitionautorefname}{Definition} % autoref-Name für definition

\renewcommand{\qedsymbol}{$\blacksquare$} % Schwarzes Quardrat als Symbol für: q. e. d.
\renewenvironment{proof}[1][\proofname]{{\bfseries #1:}~}{\qed} % "Beweise:" in Fettdruck

% *************************** PSEUDOCODE SETUP ********************************
\floatstyle{boxed}                        % Rahmen für pseudocode-Umgebung
\newfloat{pseudocode}{htbp}{lop}[section] % Definieren pseudocode-Umgebung
\floatname{pseudocode}{Pseudocode}        % Beschrifte pseudocode-Umgebung mit "Pseudocode"

\newcommand{\listofpseudocodename}{Pseudocodeverzeichnis}
\newcommand{\listofpseudocode}{\listof{pseudocode}{\listofpseudocodename}}
\providecommand*{\pseudocodeautorefname}{Pseudocode}

% ******************************* PAGE SETUP **********************************
\textheight23cm
\textwidth14cm
\voffset0cm
\topskip0cm
\topmargin-1.2cm
\headheight1.0cm
\headsep1.5cm
\oddsidemargin1.0cm
\evensidemargin1.0cm
\renewcommand{\baselinestretch}{1.4} 
\widowpenalty=300
\clubpenalty=300