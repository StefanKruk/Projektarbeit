% **************************** PACKAGE SETUP *******************************
\usepackage[ngerman]{babel}          % Lokalisierung von Typographie, Silbentrennung, etc.

\usepackage{ucs}                     % Erweiterte Unterstützung von UTF-8-Kodierung
\usepackage[utf8x]{inputenc}         % Unterstützung von UTF-8 in Eingabe-Dateien
\usepackage[T1]{fontenc}             % Zeichensatzkodierung von LaTeX (Cork-Kodierung)
\usepackage{helvet,courier,mathptmx} % Verwendete Schriftarten

\usepackage{amsmath}                 % Mathematische Infrastruktur für LaTeX der AMS
\usepackage{amsfonts}                % Mathematische Schriftarten
\usepackage{amssymb}                 % Mathematische Symbole
\usepackage{amsthm}                  % Erweiterung der Theorem-Umgebungen

\usepackage{fancyhdr}                % Erweiterte Konfiguration von Kopf/Fußzeile



\usepackage{hyperref}                % Querverweise, Hyperlink, pdf-Konfiguration, etc.

\usepackage{float}                   % Selbstdefinierte Floating-Umbgebungen
\usepackage{tabularx}                % Tabellen mit einstellbarer Spaltenbreite
\usepackage[labelfont=bf]{caption}   % Anpassen der Abbildungs- und Tabellenbeschriftungen

\usepackage{algpseudocode}           % Algorithmen als Pseudocode (basiert auf algorithmicx)
\usepackage{listings}                % Quellcode-Satz (z.B. mit Syntax-Hervorhebung)

\usepackage{graphicx}                % Erweiterte Unterstützung von Graphiken
\usepackage{textpos}                 % Beliebig platzierte Textboxen
\usepackage{xcolor}                  % TeX-Engine-unabhängige Definition von Farben

\usepackage[numbers]{natbib}   		 % Weiter Optionen für die Bibliographie

\usepackage{setspace}
\usepackage{ellipsis}	% Korrigiert den Wei�raum um Auslassungspunkte
\usepackage{placeins} 
\usepackage{tikz}