% ***************************** MAIN FILE **********************************

\documentclass[12pt]{report}           % Art des zu erstellenden Dokuments
% bei zweiseitigem Druck twoside-Option oder book-Klasse verwenden

% ****************************** PREAMBLE **********************************
\input{preamble/packages}

% ****************************** TOP MATTER ***********************************
\renewcommand{\author}{Stefan Kruk}           % Name
\newcommand{\dateOfBirth}{14.08.1992}           % Geburtsdatum
\newcommand{\matrNumber}{7084972}              % Matrikelnummer
\newcommand{\studycourse}{Softwaretechnik (Dual)}           % Studiengang

\newcommand{\supervisor}{Prof. Dr. Johannes Ecke-Schüth} % Betreuer
\newcommand{\institution}{Fachhochschule Dortmund\xspace} % Hochschule
\newcommand{\faculty}{Informatik\xspace}               % Fachbereich
\newcommand{\toponym}{Dortmund}                 % Ort

\newcommand{\subject}{Projektarbeit}  % Art/Thema der Arbeit
\newcommand{\titel}{Unterschiede zwischen SOA und Microservices} % Titel der Arbeit
%\newcommand{\subtitel}{Zweizeiliger Untertitel\\sofern vorhanden} % Untertitel
\newcommand{\degree}{Bachelor/Master of Art\xspace} % Angestrebter Titel (nur bei Abschlussarbeiten, sonst leer lassen/auskommentieren)

\newcommand{\keywords}{Projektarbeit, SOA, Microservices, Informatik, {FH Dortmund}} % Stichworte (durch Komma getrennt)

\input{preamble/setup}

% ******************************** MACROS *************************************
\newcommand{\RR}{\mathbf{R}}
\newcommand{\NN}{\mathbf{N}}
\newcommand{\QQ}{\mathbf{Q}}
\newcommand{\ZZ}{\mathbf{Z}}
\newcommand{\CC}{\mathbf{C}}
\newcommand{\SOA}{Service-orientierte Architektur}
\newcommand{\ebay}{eBay}
\newcommand{\secref}[1]{\ref{#1} \nameref{#1}}
\newcommand{\gmbh}{OnlineCommerceShop GmbH}
\newcommand{\quelle}{\footnotesize Quelle: }
\newcommand{\ci}{Continuous Integration}
\newcommand{\cd}{Continuous Delivery}

%%%%%%%%%%%%%%%%%%%%%%%%%%%%%%%%%%%%%%%%%%%%%%%%%%%%%%%%%%%%%%%%%%%%%%%%%%%%%%%
% ****************************** Glossarie ************************************
%%%%%%%%%%%%%%%%%%%%%%%%%%%%%%%%%%%%%%%%%%%%%%%%%%%%%%%%%%%%%%%%%%%%%%%%%%%%%%%

%Befehle für Symbole
\newglossaryentry{symb:Pi}{
    name=$\pi$,
    description={Die Kreiszahl.},
    sort=symbolpi, type=symbolslist
}
\newglossaryentry{symb:Phi}{
    name=$\varphi$,
    description={Ein beliebiger Winkel.},
    sort=symbolphi, type=symbolslist
}
\newglossaryentry{symb:Lambda}{
    name=$\lambda$,
    description={Eine beliebige Zahl, mit der der nachfolgende Ausdruck
        multipliziert wird.},
    sort=symbollambda, type=symbolslist
}

%Befehle für Glossar
\newglossaryentry{glos:AD}{
    name=Active Directory,
    description={Active Directory ist in einem Windows 2000/" "Windows
        Server 2003-Netzwerk der Verzeichnisdienst, der die zentrale
        Organisation und Verwaltung aller Netzwerkressourcen erlaubt. Es
        ermöglicht den Benutzern über eine einzige zentrale Anmeldung den
        Zugriff auf alle Ressourcen und den Administratoren die zentral
        organisierte Verwaltung, transparent von der Netzwerktopologie und
        den eingesetzten Netzwerkprotokollen. Das dafür benötigte
        Betriebssystem ist entweder Windows 2000 Server oder
        Windows Server 2003, welches auf dem zentralen
        Domänencontroller installiert wird. Dieser hält alle Daten des
        Active Directory vor, wie z.B. Benutzernamen und
        Kennwörter.}
}
\newglossaryentry{glos:AntwD}{name=Antwortdatei, description={Informationen zum
        Installieren einer Anwendung oder des Betriebssystems.}}

%Befehle für Abkürzungen
\newacronym{MS}{MS}{Microsoft}
\newacronym{CD}{CD}{Compact Disc}
%Eine Abkürzung mit Glossareintrag
\newacronym{AD}{AD}{Active Directory\protect\glsadd{glos:AD}}

%%%%%%%%%%%%%%%%%%%%%%%%%%%%%%%%%%%%%%%%%%%%%%%%%%%%%%%%%%%%%%%%%%%%%%%%%%%%%%%
% ************************** BEGINN OF DOCUMENT *******************************
%%%%%%%%%%%%%%%%%%%%%%%%%%%%%%%%%%%%%%%%%%%%%%%%%%%%%%%%%%%%%%%%%%%%%%%%%%%%%%%
\begin{document}

\input{misc/titlepage}

% ***************************** FRONT MATTER **********************************
\setcounter{page}{1}
\pagenumbering{roman}

\input{misc/tables}

\chapter*{Überblick}
\section*{Kurzfassung}
In dieser Arbeit werden die Vor- und Nachteile von \SOA\ und Microservices erörtert und anschließend miteinander verglichen. Dafür wird eine Ausgangssituation beschrieben und die Problematiken analysiert, aus welcher sich die zu untersuchende Fragestellung bildet. Nachdem beide Paradigmen getrennt erläutert wurden, werden diese miteinander verglichen.

Nach dem Vergleich wird auf die  zu untersuchenden Fragestellungen eingegangen. Es soll insbesondere untersucht werden, in wie weit die Paradigmen für die Lösung dieser Fragestellungen geeignet sind. Abschließend wird ein Fazit aus den gewonnenen Ergebnissen gezogen und ein Ausblick auf folgende Arbeiten beschrieben.

\section*{Abstract}
In this Projekt will be the advantages and disadvantages of SOA and Microservices described. After that both paradigm models will be compared. For this there will be described a initial situation and the problems will be analyzed from which it will be formulated questions. After both paradigm were seperately explained, they will be compared.

After comparison the to analyzed questions will be analyzed. It should in particular analyzed if one or both paradigm are solutions for the questions. Finally there will be drawn a conclusion from the results and a outlook for following Projekts.

% ******************************* Glossar ************************************
%Alle Glossare ausgeben
%\printglossaries

%Glossar ausgeben
\printglossary[style=altlist,title=Glossar]


%Abkürzungen ausgeben
\deftranslation[to=German]{Acronyms}{Abkürzungsverzeichnis}
\printglossary[type=\acronymtype,style=long]

%Symbole ausgeben
\printglossary[type=symbolslist,style=long]
\newpage

% ***************************** MAIN MATTER ***********************************
\pagestyle{fancy}
\pagenumbering{arabic}

\chapter{Einleitung}
\label{chap:einleitung}

\section{Motivation}
In fast jedem Unternehmen wird Software eingesetzt und oft wird neue Software eingeführt oder, was deutlich seltener der Fall ist, alte Software durch neue ausgetauscht. In jedem Fall muss sich die neue Software in die bestehenden Prozesse und Architekturen integrieren lassen, damit sie genutzt werden kann.

Kleinere Unternehmen haben es daher deutlich einfacher, denn sie besitzen meistens nur einen Zentralen Server mit wenig Software. Große Unternehmen hingegen haben es da deutlich schwerer. Deren Infrastruktur basiert nicht auf einen Server, sondern auf ganze Rechenzentren weltweit, in denen die Server stehen, die sie nutzen. Damit Unternehmenssoftware miteinander, anstatt gegen- oder parallel zueinander arbeiten können, sollten Gedanken darüber gemacht werden, wie Software in dem Unternehmen aufgebaut sein soll. Große Systeme sind meist heterogen und haben gewollt oder ungewollt Redundanzen. Das kann anfangen mit der Berechnung eines bestimmten, unternehmensweiten, Zinssatzes, bis hin zu ganzen Prozessen welche doppelt in Software abgebildet werden. Dies verkompliziert die Wartung und Optimierung bestimmter Prozesse enorm.

Nehmen wir das Beispiel der Zinsberechnung. Soll diese Berechnung angepasst oder verändert werden, muss dies in jeder Software geschehen, welche diese Zinsen berechnet. Wählt man jedoch eine geeignete Softwarearchitektur, kann dieses vorgehen deutlich vereinfacht werden. Eine Lösung könnte sein, Microservices zu nutzen. Eine andere könnten \SOA en sein. Bei beiden Modellen basiert das vorgehen darauf, dass kleine Services vorhanden sind, welche bestimmte aufgaben übernehmen und die entsprechenden Programme auf diese Services zurückgreifen, anstatt sie selber zu implementieren.

\section{Ausgangssituation}\label{sec:ausgangssituation}
Folgende Institute sind gegeben:
\begin{itemize}
    \item Einwohnermeldeamt
    \item Standesamt
    \item KFZ-Zulassungsstelle
    \item Polizei
\end{itemize}
Diese Institute müssen Möglichkeiten haben, um bestimmte Daten voneinander abzufragen. Zum Beispiel muss der Fahrzeughalter und sein derzeitiger Wohnort bei einem vergehen mit dem PKW ermittelt werden können. Dazu muss die Polizei zunächst einmal mit Hilfe des Kennzeichens den Fahrzeughalter über die KFZ-Zulassungsstelle ermitteln. Anschließend muss dieser über das Einwohnermeldeamt ausfindig gemacht werden.

Ein weiteres Beispiel: Zwei Personen Heiraten Standesamtlich. Somit ändert sich der Nachnamen von mindestens einer Person. Das Standesamt muss also das Einwohnermeldeamt darüber informieren, dass sich der Nachname dieser Person/en ändert.
\section{Vorgehen}
Zunächst werden die Grundlagen von Microservices und \SOA , sowie monolithischen Architekturen erläutert. Darauf aufbauend wird der Kontext, welcher im Kapitel \secref{sec:ausgangssituation} beschrieben wurde, mit dem Microservice-Modell und dem \SOA\ Modell implementiert und beide Architekturen miteinander verglichen.

Abschließend wird noch einmal das Ziel dieser Arbeit erläutert. Danach werden die Ergebnisse präsentiert, bewertet und ein Fazit daraus gezogen. Zuletzt wird ein Ausblick über die weiteren Möglichkeiten der Architekturen erläutert.

\chapter{Problemanalyse}
\label{chap:analyse}
Jedes Internet-Unternehmen fängt klein an und wächst mit der Zeit. In dieser Zeit muss das Unternehmen viele Probleme bewältigen. Es muss die Produkte, welches das Unternehmen verkauft, weiterentwickeln und auf die Bedürfnisse der Nutzer reagieren. Dabei ist die Wahl des Architekturmodells entscheidend.

\section{Herausforderungen}
\label{sec:herausforderung}
Oft passiert es, das ein Unternehmen große Vorstellungen von dem Unternehmens-Ziel hat. Dies führt häufig dazu, dass Software entwickelt wird, welches weit über den aktuellen Anforderungen hinaus gehen. Man möchte damit verhindern, Software an einem späteren Zeitpunkt neu zu entwickeln oder austauschen zu müssen. Jedoch kann dies zu großen Problemen führen sobald das Unternehmen wächst und den am Anfang genannten Vorstellungen näher kommt. In dieser Phase entstehen meistens Probleme, welche vorher nicht berücksichtigt worden sind, weil niemand sie kannte. Dadurch muss die vorhandene Software, welche eigentlich für dieses Szenario ausgelegt war, geändert werden.

Wie es bereits in \secref{sec:motivation} erläutert wurde, ist die Zeitspanne es Time-to-Market für ein Unternehmen von äußerster Bedeutung. Damit diese Zeitspanne möglichst gering ist, bestehen besondere Anforderungen an Software. Vor allem, wenn es um die Einführung neuer Funktionen geht entstehen meistens lange TTM-Zeiten. 

\section[Beispiel]{Hypothetisches Beispiel in Anlehnung an ein reales Problem}
\label{sec:beispielEbay}
%Als Beispiel dient ein fiktives Unternehmen namens "OnlinePlattform GmbH". Das Unternehmen ist ein Modernes und Aktives Internet-Unternehmen, welches seit 1990 eine Plattform zum verkauf und Versteigern von Produkten anbietet. Die Software hinter der Plattform wurde anfänglich als Monolitsche Perl Applikation geschrieben.
Ein Modernes und aktives Internet-Unternehmen ist \ebay und darf man der Seite \cite{highscalability} glauben, so startet die Seite 1995 als Monolithische Perl Applikation. Selbst wenn diese Informationen nicht auf \ebay zutreffen sollten, so dient es uns doch als gutes Beispiel einer realen Problemstellung.

\ebay\ wurde 1995 von Pierre Omidyar unter dem Namen \textit{ActionWeb} gegründet und wurde 1997 in \ebay umbenannt. \ebay\ wurde wie oben schon angesprochen als Monolithische Perl Anwendung implementiert (siehe \cite{wiki:ebay}). Mit der Steigerung der Reichweite und der täglichen Benutzung, hatte man sich dann aber dazu entschlossen auf C++ als Code-Basis umzusteigen und die Seite mit CGI zu implementieren. Mittlerweile war \ebay ein großes und sich rasant entwickelndes Unternehmen. Das bedeutet aber auch, dass \ebay ständig auf das Verhalten der Nutzer reagieren und sich anpassen muss. Man versuchte also eine Monolithische Applikation mit der Fähigkeit auszustatten, auf Änderungen schnell zu reagieren, implementieren und deployen. Aber ein Monolith zu deployen, bedeutet, entweder die gesamte Infrastruktur für Wartungszwecke offline zu nehmen und die Applikation neu zu deployen, oder Server im Parallel betrieb laufen zu lassen und jeden neuen Traffic auf die neue Version zu routen. Die letzte Methode bietet jedoch einige Schwierigkeiten, denn es könnten Änderungen eingebaut worden sein, welche im Konflikt mit der alten Version stehen. Dann muss in jedem Fall die erste Variante gewählt werden und die gesamte Infrastruktur offline genommen werden.

Beide Varianten der Änderungen sind jedoch zeitaufwendig und  schwierig, denn nicht nur das deployen könnte Probleme bereiten, sondern auch die darauf folgende Ausführung des Programms. Ändert man Code in Monolithischen Applikationen kann das auch Auswirkungen auf bestehende Teile des Codes haben, welche vorher, ohne Probleme, funktioniert haben. Werden Tests vernachlässigt oder wird nicht ausreichend getestet, kann es leicht passieren, dass sich ungewollt Fehler einschleichen, wodurch dann eine Version in betrieb genommen wird, welche Fehler enthält. Darauf folgend müssten diese wieder behoben werden und die Anwendung erneut deployed werden.

Im Falle einer Monolithischen Architektur bedeutet das viele Änderungen und neue Features. Oft passiert es daher, dass Code Stücke zurückbleiben, welche nicht mehr benötigt werden. Irgendwann ist die Applikation daher so groß, dass sie nicht mehr Wartbar ist und neue Features nur noch schwer zu implementieren sind. Entstehen Fehler in solch einer Anwendung ist es um so schwerer diese zu finden und zu beheben. Schließlich hat sich \ebay entschlossen ihre "Anwendung" in Java neu zu implementieren. Dieses mal jedoch mit dem Hintergrund einer leicht erweiterbaren und wartbaren Architektur.

\section{Die zu untersuchende Fragestellung}
\label{sec:dasProblem}
Das Problem besteht also darin, "eine Anwendung" flexibel und einfach erweiterbar zu gestalten, damit ein Unternehmen schnell auf Änderungen und die veränderten Bedürfnisse der Nutzer reagieren kann. Damit solch eine Software ebenfalls gut Wartbar ist, sollten Redundanzen möglichst vermieden werden. Hier steigen wir in Modularität ein, denn damit eine Software möglichst einfach Wartbar ist, sollte jeder Code mit einem bestimmten Kontext in ein eigenes Modul gepackt werden. So können zum Beispiel alle Codestücke einer bestimmten Berechnung in ein Modul gepackt werden, wodurch diese dann nur an einer zentralen Stellen geändert werden muss. Wenn die Module jedoch auch von verschiedenen Applikationen genutzt werden, müssen diese in Libraries ausgelagert werden. Auch hier hat man wieder nur eine zentrale Stelle, an der Änderungen vorgenommen werden müssen, um die Berechnung anzupassen oder zu ändern. Jedoch besteht hier das Problem, dass wenn Änderungen durchgeführt werden, diese noch lange nicht in jeder laufenden Applikation zu finden sind. Dazu müssen diese nämlich erst einmal neu gebaut und deployed werden, damit diese Produktiv werden.

Um dieses Vorgehen zu vereinfachen hat man sich dazu entschieden, diese Libraries in eigene Applikationen (Services) zu verpacken und diese über eine Schnittstelle anzubieten. Das sorgt dafür, dass jede Software, welche dieses Modul benötigt, sie nicht mehr eigenständig implementieren muss, sondern den dafür vorgesehenen Service aufrufe kann, um die nötigen Informationen zu erhalten. Diese Services werden auch als Microservices bezeichnet, da sie nur einem bestimmten Zweck dienen, dafür ihre Aufgabe aber besonders gut erledigen. Hierbei kann man unter anderem von einer \SOA\ (SOA) sprechen. Es sollte jedoch darauf geachtet werden, dass weder zu viel, noch zu wenig in ein Service gepackt wird. Welche Größe genau richtig ist, wird im Kapitel \secref{chap:grundlagen} weiter erläutert.



% Das gesetzt von Conway [Seite 39 EWolff2016:Microservices]
\chapter{Grundlagen}
\label{chap:grundlagen}
Software zu entwickeln ist nicht immer einfach. Umso größer diese ist, umso mehr Probleme können auftreten. Es bedarf einer genauen Planung und Verständnis von Infrastruktur um eine Software mit allen Anforderungen zufriedenstellend zu implementieren. 

Vor allem wenn es um die Weiterentwicklung und Wartung von Software geht, können große Probleme auftreten. Wurde die Architektur nicht gut gewählt oder schlecht umgesetzt, kann es das weitere Vorgehen stark beeinträchtigen, bis hin zum unmöglich machen. Es wurden daher Software-Architekturen entwickelt, welche flexibel und einfacher zu ändern sind. Außerdem kann in diesen Architekturen neue Funktionen deutlich schneller hinzugefügt werden.

\section{Architektur}
\label{sec:architektur}
Microservice-Architekturen und \SOA (SOA) Architekturen können, wenn sie richtig angewendet werden, sehr flexibel und schnell änderbar sein. Beide Architekturen zielen, wie der Name schon sagt, auf eigenständige Services ab, welche durch verschiedene Kommunikationskanäle miteinander kommunizieren und dadurch die gewünschten Geschäftsprozesse abbilden. "Ein Programm soll nur eine Aufgabe erledigen, und das soll es gut machen" \cite[S. 2]{EWolff2015:ContinuouosDelivery}. Anstatt eine einzige große Anwendung ein zu setzten, setzt man auf viele kleine, verteile, autarke Anwendungen, welche jeweils Schnittstellen nach außen hin anbieten damit der Service genutzt werden kann. Diese Schnittstellen können unter anderem durch REST-HTTP angeboten werden.
Durch die verteilten Anwendungen funktioniert das System auch dann noch, wenn einzelne Services nicht verfügbar sind, jedoch bringt es ebenfalls die typischen Probleme von Verteilten Anwendungen mit sich, welche in \secref{chap:fallstudie} noch genauer erläutert werden.
\\\\
\textbf{Services kann man in drei Kategorien einteilen:}
\begin{description}
    \item[Producer] Ein Service der etwas Produziert oder auf eine Anfrage reagiert. Das reicht von Daten aus einer Datenbank extrahieren bis hin zu komplexen Berechnungen.
    \item[Consumer] Ein Service oder eine Anwendung, welche einen oder mehrere Produzierende Services verwendet und entweder weiterverarbeitet oder ausgibt. Im Falle der Weiterverarbeitung ist ein Consumer ebenfalls ein Producer sein.
    \item[Self-Contained System (SCS)] "\frqq Microservice mit UI\flqq\ oder \frqq Self-Contained System\flqq\ wie es Stefan Tilkov nennt, sind in sich abgeschlossene Systeme. [..] Sie enthalten eine UI und sollten möglichst nicht mit anderen SCS kommunizieren." \cite[vgl S. 55]{EWolff2016:Microservices}. SCS sind jedoch nur im Microservice und nicht im SOA Umfeld zu finden. Warum wird in \secref{chap:fallstudie} weiter erläutert.
\end{description}

\subsection{Das Gesetzt von Conway}
\label{subsec:conway}
Wie in \cite[S. 39 ff.]{EWolff2016:Microservices} beschrieben, stammt das Gesetzt von dem amerikanischen Informatiker Melvin Conway und besagt:
\begin{center}
    \textit{Organisationen, die Systeme designen, können nur solche Designs entwerfen, welche die Kommunikationsstruktur dieser Organisationen abbilden.}
\end{center}
"Conway möchte damit ausdrücken, dass die internen Kommunikationswege wichtig bei der Planung der Architektur ist. Jedes Team innerhalb einer Organisation trägt bei der Entwicklung der Architektur bei. Wird eine Schnittstelle zwischen zwei Teams benötigt, so müssen diese Teams auch kommunizieren können. Dabei müssen Kommunikationswege nicht immer offiziell sein. Oft gibt es informelle Kommunikationsstrukturen, die ebenfalls in diesem Kontext betrachtet werden können." \cite[vg. S. 39]{EWolff2016:Microservices}

\section{Werkzeuge}
\label{sec:werkzeuge}
Anders als bei Monolithischen-Projekten, bestehen Service-Orientierte-Projekte aus vielen kleinen Services, die alle einzeln verwaltet werden müssen. Das sie laufen reicht nicht aus, sie müssen auch geupdated und gewartet werden.

Im folgenden sollen daher einige Werkzeuge vorgestellt werden, welche den Umgang mit Services vereinfachen. Dabei geht es um Werkzeuge, welche Entwickler vom Deployment bis zur Wartung der Applikationen unterstützen.

\subsection{Jenkins}
\label{subsec:jenkins}
Jenkins ist ein in Java geschriebenes, webbasiertes Software-System, für \ci\ (CI) \ \cd\ (CD). Es kann unter anderem durch Plugins um weitere Funktionen erweitert werden. Mit Hilfe von Jenkins sollen das Bauen, Testen und Ausliefern automatisiert werden.\footnote{\cite[vgl. S. 98 ff.]{EWolff2015:ContinuouosDelivery}}


\subsection{Puppet}
\label{subsec:puppet}
Puppet ist ein Systemkonfigurationswerkzeug. Es soll durch Konfigurationsdateien eine Standardisierte Installation von Software ermöglichen. Über ein Master-Slave System können diese dann via Netzwerk auf mehreren Computern/Servern verteilt werden.

\subsection{Docker}
\label{subsec:docker}
Docker soll die Auslieferung von Anwendungen vereinfachen, in dem die Anwendung in ein Container isoliert wird. Die so entstehende Datei wird Image genannt. Zum Ausliefern der Anwendung reicht es, wenn das Image an den jeweiligen Bestimmungsort kopiert und dort ausgeführt wird. Das kann zum Beispiel durch Puppet geschehen. Dies ermöglicht außerdem, dass eine Anwendung auf verschiedenen Servern exakt gleich ausgeführt wird.
Docker setzt dabei auf Linux Container. Dadurch ist nur eine Instanz des Kernels im Speicher, jedoch sind die Prozesse, Benutzer, Dateisystem und Netzwerk von einander getrennt. Das ermöglicht außerdem, dass mehrere Docker Images zusammen arbeiten.\footnote{\cite[vgl. S. 53 ff.]{EWolff2015:ContinuouosDelivery}}

\subsection{Vagrant}
\label{subesec:vagrant}
Mit Hilfe von Vagrant kann man einfach und schnell Virtuelle Maschinen standardisiert aufsetzten. Dabei nutzt das Werkzeug fertige Betriebssystem Images, wie sie zum Beispiel mit VirtualBox erstellt werden können. Die Virtuelle Maschine wird durch das sogenannte Vagrantfile konfiguriert. Innerhalb dieser Datei kann zum Beispiel mit Puppet zusätzliche Software installiert werden. Dadurch kann schnell ein neues Produktiv- oder Testsystem aufgebaut werden.
\footnote{\cite[vgl. S. 49 ff.]{EWolff2015:ContinuouosDelivery}}

\subsection{Swagger}
\label{subsec:swagger}
Die bisherigen Werkzeuge dienten dazu, Services auszuliefern, aber sie müssen, wie bereits weiter oben erwähnt, Schnittstellen anbieten, damit sie genutzt werden können. Diese müssen jedoch auch dokumentiert sein, damit ein Entwickler weiß, welche Schnittstelle er wie anzusprechen hat. Durch Swagger ist eine einfache und zentrale Dokumentation von REST-HTTP Schnittstellen möglich.

%Zookeeper, etcd
%Spring Cloud Config
\subsection{Service Discovery - Spring}
\label{subsec:ServiceDiscovery}
Mit Swagger haben wir die Schnittstellen dokumentiert, jedoch kann es bei vielen Services problematisch werden, den richtigen zu finden. Dafür sind sogenannte Service Discovery Werkzeuge wie Spring Eureka von Netflix zuständig. Sie können außerdem dafür sorgen, dass ein Programm automatisch einen anderen Server bekommt, wenn ein genutzter Service ausfallen sollte. \footnote{\cite[vgl. S. 326 ff.]{EWolff2016:Microservices}}

\subsection{ELK (ElasticSearch, LogStash, Kibana)}
\label{subsec:elk}
Durch Service-Orientierte Systeme gibt es viele verteilte Anwendungen. Um sie zu Warten und Fehler beheben zu können, müssen die Protokoll Dateien ausgewertet werden. Da jedoch die Services im Netzwerk verteilt sind, sind auch die Protokolle verteilt. Durch den ELK-Stack ist ein einheitliches Logging möglich.
Mit Hilfe von \textit{Logstash} können die Log-Dateien im von Servern im Netzwerk eingesammelt und mit \textit{Elasticsearch} zentral gespeichert werden. \textit{Kibana} ist eine Weboberfläche, mit der die gespeicherten Protokoll Daten in ein für Menschen leserliches Format angezeigt und ausgewertet werden kann.
\footnote{\cite[vgl. S. 244 ff.]{EWolff2016:Microservices}}

\chapter{SOA}
\label{chap:soa}
Der Begriff SOA ist nicht eindeutig definiert. Je nachdem welche Person man in einem Unternehmen fragt, erhält man unter Umständen eine komplett andere Definition. \frqq Die Meisten Definitionen stimmen jedoch zum größten Teil überein und stehen nicht im Konflikt mit einander.\flqq\cite[vgl. Seite 6]{100QA}

Für einen Kaufmann ist SOA etwas anderes als für einen Analysten, damit SOA jedoch verstanden werden kann, werden zunächst einmal einige Definitionen nach \cite{100QA}\ genannt:
\begin{enumerate}
       \item \frqq To the chief information officer (CIO), SOA is a journey that
       promises to reduce the lifetime cost of the application portfolio [...].\flqq \cite[vgl. Seite 6]{100QA}
    
       \item \frqq To the business executive, SOA is a set of services that can be exposed to their customers, partners, and other parts of the organization. Business capabilities, function, and business logic can be combined and recombined to serve the needs of the business now and tomorrow. Applications serve the business because they are composed
       of services that can be quickly modified or redeployed in new
       business contexts, allowing the business to quickly respond to changing
       customer needs, business opportunities, and market conditions.\flqq \cite[vgl. Seite 6]{100QA}
       
       \item \frqq To the business analyst, SOA is a way of unlocking value, because business processes are no longer locked in application silos. Applications no longer operate as inhibitors to changing business needs.\flqq \cite[vgl. Seite 6]{100QA}
       
       \item \frqq To the chief architect or enterprise architect, SOA is a means to
       create dynamic, highly configurable and collaborative applications
       built for change. SOA reduces IT complexity and rigidity. SOA becomes the solution to stop the gradual entropy that makes applications
       brittle and difficult to change. SOA reduces lead times and costs
       because reduced complexity makes modifying and testing applications
       easier when they are structured using services.\flqq \cite[vgl. Seite ]{100QA}
\end{enumerate}
Jeder der genannten Rollen hat eine eigene klare Definition von dem was SOA ist. Jeder der Definitionen ist jedoch nur ein Teil dessen für was SOA verwendet werden kann. Denn SOA ist eine Herangehensweise und kein festes Modell.

\section{Grundlagen}
\label{sec:Grundlagen}
Das Ziel von SOA ist nicht die Entwicklung zu vereinfachen oder voran zu bringen. SOA soll die Unternehmensweiten Geschäftsprozesse standardisieren und vereinheitlichen. Aus diesem Grund sollte SOA nicht als Modell, sondern als Herangehensweise verstanden werden.
\\\\
Bei der Vereinheitlichung und Standardisierung spielt die Wertschöpfungskette eine wichtige Rolle, da diese die Geschäftsprozesse miteinander verbindet.
\\\\
Bei der Vereinigung verschiedener Geschäftsprozesse spielt die Kommunikation ebenfalls eine entscheidende Rolle. Wie in Kapitel \secref{subsec:conway}\ erwähnt, können nur solche Systeme entworfen werden, welche die Kommunikationsstrukturen der Organisation abbilden. Daher muss, wenn man SOA verwenden möchte, die nötigen Kommunikationswege geschaffen werden, um eine ordnungsgemäße und standardisierte Kommunikation zu gewährleisten.
\\\\
SOA soll zudem dabei helfen Geschäftsprozesse und Geschäftskomponenten in ein bestehendes Unternehmen einzubinden. Als Beispiel die \textit{Auktionen GmbH}\ übernimmt das Unternehmen \textit{Handel GmbH}. Beide Unternehmen besitzen zwar ähnliche Prozesse und Komponenten, können jedoch nicht ohne weiteres in die bestehende Struktur übernommen werden. Mit Hilfe von SOA soll dies vereinfacht werden.

\subsection{Business und IT}
\label{subsec:BusinessAndIT}
Ein Zentraler Bestandteil von SOA ist die IT. IT darf nie zum selbst Zweck existieren. Sie wird immer zur Unterstützung der Geschäftsprozesse eingesetzt. Umso mehr Geschäftsprozesse existieren, umso mehr Software existiert in einem Unternehmen.
\\\\
Oft existieren bereits verschiedene Anwendungen wie ERP- oder COBOL-Systeme. Mit SOA soll dafür gesorgt werden, das diese Systeme möglichst effizient miteinander arbeiten können.
\begin{quotation}
    \frqq The complexity of the technology infrastructure at many companies in the financial services sector makes it very hard to leverage IT services in a coordinated way across the enterprise. Many large companies have either merged or acquired other very large companies resulting in the integration of new business units with very different work cultures and widely different information infrastructurse. The need to be able to trust and understand the information about the business across its many disaggregated parts has been a prime motivator for change in the IT infrastructure at these companies.\flqq \cite[S. 17]{SOAForDummies}
\end{quotation}

Mit SOA wird die IT-Infrastruktur eines Unternehmens in zwei Teile geteilt. Auf der einen Seite existiert die Geschäftsschicht mit der Geschäftslogik und auf der anderen Seite die IT-Schicht, welche die Computing-Ressourcen verwaltet. Durch diesen Aufbau ist es nicht nötig, das ein Business Manager die IT-Schicht verstehen muss.
\\\
In der Geschäftsschicht sind nur Dienste, mit denen Kunden, Lieferanten und Business Partner interagieren. Diese Personen benötigen, genauso wie ein Business Manager, keine Wissen darüber, was in der IT-Schicht existiert oder wie diese aufgebaut ist. Andersherum sind in der IT-Schicht nur Dienste und Applikationen vorhanden, wofür die IT-Abteilung zuständig ist.
\\\\
Damit diese Schichtentrennung funktioniert, wird darauf geachtet, dass in der Geschäftsschicht möglichst wenig Komplexität nach außen sichtbar ist.

\subsection{Unternehmens Komponenten}
\label{subsec:UnternehmensKomponenten}
Dazu müssen zunächst einmal alle Komponenten eines Unternehmens identifiziert werden. Die nachstehende Abbildung (\ref{fig:UnternehmensKomponenten}) zeigt ein Beispiel dieser Identifizierung:

\begin{figure}[htb]
    \centering 
    \includegraphics[width=\linewidth]{content/images/UnternehmensKomponenten}\
    \quelle\url{http://www.jot.fm/issues/issue_2008_05/column5/}
    \caption[Unternehmens Komponenten]{Unternehmens Komponenten\\}
    \label{fig:UnternehmensKomponenten}  
\end{figure} 
\newpage
Aus diesen Komponenten müssen nun die Geschäftsprozesse identifiziert werden, beginnend mit den wichtigsten. Daraus ergeben sich anschließend die Komponenten, welche unter einander kommunizieren müssen. Die in der Abbildung Rot dargestellten Komponenten sind schließlich das Resultat aus der Analyse. Hat man die Komponenten über Kommunikationswege verbunden, werden die nächsten Geschäftsprozesse identifiziert. Diese Prozedur wird solange wiederholt, bis alle Komponenten mit einander verbunden sind.

\section{Architektur}
\label{sec:SoaArchitektur}
Die Architektur in einem SOA-System unterliegt dem Paradigma einer "`Service-orientierten Architektur"'. Mit diesem Paradigma ist der Begriff \textit{Dienst} verbunden. Dieser ist im Falle von SOA jedoch zunächst nicht richtig, denn Anwendungen wie ERP- oder COBOL-Systeme sind eigenständige Anwendungen und keine Dienste.
\\\\
Damit eine ERP-Anwendungen ihre Funktionalitäten bereitstellen kann, muss ein Adapter erstellt werden. Dieser Adapter stellt die Funktionalität in Form von Schnittstellen bereit und kann als Dienst bereitgestellt werden. Der Adapter übernimmt dabei Aufgabe wie Fehler abzufangen und die Daten in angemessener Weise aufzuarbeiten, damit diese über die jeweilige Schnittstelle abgerufen werden können, sowie die eigentliche Anwendung zu steuern.
\\\\
Schnittstellen können dabei HTTP-Rest oder SOAP sein. Es ist jedoch auch Möglich weitere Schnittstellentechnologien zu verwenden, wie RMI. Genauso kann es möglich sein "`Message Oriented Middleware"' zu verwenden, um Informationen bereit zu stellen.

\section{Enterprise-Service Bus - ESB}
\label{sec:esb}
Der Begriff "`Enterprise-Service Bus (ESB)"' wird oft im Zusammenhang mit SOA verwendet. Der ESB ist jedoch kein Teil von SOA, sondern nur ein Mittel zum Zweck, um die Kommunikation zu vereinheitlichen. Der Enterprise-Service Bus dient dabei als Nachrichten Komponente, welcher die Nachrichten entgegen nimmt und an die jeweiligen Empfänger weiterleitet, sowie als Transformator von Nachrichten, um Interoperabilität zu schaffen. Außerdem kann der ESB häufig als \textit{Service Discovery} eingesetzt werden. Jedoch kann diese Aufgabe auch von einer eigenständige Anwendung, wie \textit{Zookeeper} oder \textit{Consul}, übernommen werden.
\\\\
Grundsätzlich wird ein ESB eingesetzt, um an einer zentralen Stelle die Kommunikation zu steuern und die Geschäftsprozesse abzubilden. Am einfachsten lässt sich das an einem Beispiel erklären.
\\
Man Stelle sich eine Banking-Anwendung vor, an der man sich anmeldet. Es werden daraufhin folgende Informationen angezeigt:

\begin{enumerate}
    \item Name
    \item Kontostand
    \item EC- und Kreditkarten
    \item Liste der Aktienfonds
\end{enumerate}

Jede Information stammt aus einem anderen Teil des Systems und werden von verschiedenen Anwendungen über Schnittstellen bereitgestellt. Die Schnittstellen können sich dabei von Anwendung zu Anwendung unterscheiden. So kann zum Beispiel eine Anwendung die Informationen über HTTP bereitstellen, während eine andere sie über SOAP bereitstellt. So können die Informationen zum Beispiel aus einem CRM-System (Kundenbeziehungsmanagement-System) stammen oder durch PHP oder Ruby erzeugt werden.

Anders als man jedoch vermuten würde, lässt man die Oberfläche, bzw. das System, nicht direkt mit den einzelnen Komponenten reden. Die gesamte Kommunikation läuft dabei über den \textit{Enterprise Service Bus} ab. Nachstehende Abbildung soll dies genauer erläutern:

\begin{figure}[htb]
    \centering 
    \includegraphics[width=\linewidth]{content/images/esb-ok}\
    \caption[ESB]{Enterprise Service Bus}
    \quelle\url{https://zato.io/docs/intro/esb-soa-de.html}
    \label{fig:esb}  
\end{figure}
\newpage
Benötigt eine Anwendung (in der Abbildung App genannt) Informationen, konsultiert diese zunächst den ESB. Der ESB kennt die anderen Anwendungen oder benutzt einen Dienst, welcher die Informationen bereitstellen kann, und stellt daraufhin die angeforderten Informationen bereit. Dabei ist es egal wo die Dienste liegen, solange sie über das Netzwerk ansprechbar sind.

\chapter{Microservices}
\label{chap:Microservices}

\section{Überblick}
\label{sec:überblickMicroservice}
\begin{quotation}
    \frqq Modularisierung ist nichts Neues. Schon lange werden große Systeme in kleine Module unterteilt, um Software einfacher zu erstellen, zu verstehen und weiterzuentwickeln. Das Neue: Microservices nutzen als Module einzelne Programme, die als eigene Prozesse laufen. Der Ansatz basiert auf der UNIX-Philosophie. Sie lässt sich auf drei Aspekte reduzieren:\flqq\cite[S. 2]{EWolff2016:Microservices}
\end{quotation}


\begin{itemize}
    \item Ein Programm soll nur eine Aufgabe erledigen, und das soll es gut machen.
    \item Programme sollen zusammenarbeiten können.
    \item Nutze eine universelle Schnittstelle. In UNIX sind das Textströme.
\end{itemize}
Diese Art der Aufteilung wurde schon lange von großen Unternehmen wie Amazon oder Google genutzt und wurde zu nächst \SOA (SOA) genannt. Jedoch unterscheiden sich Microservices und SOA voneinander. Daher wird SOA im Nächsten Kapitel genauer erläutert und im Kapitel \ref{chap:ergebnisse} die Ergebnisse zusammen gefasst und beide Technologien mit einander verglichen.
\\\\
Der Begriff Microservices ist nicht eindeutig definiert. Als erste Näherung dienen, nach Eberhard Wolff \cite[S. 2]{EWolff2016:Microservices}, folgende Kriterien:

\begin{itemize}
    \item Microservices sind ein Modularisierungskonzept. Sie dienen dazu. ein großes Software-System aufzuteilen - und beeinflussen die Organisation und die Software-Entwicklungsprozesse.
    \item Microservices können unabhängig von Änderungen an anderen Microservices in Produktion gebracht werden.
    \item Microservices können in unterschiedlichen Technologien implementiert sein. Es gibt keine Einschränkung auf eine bestimmte Programmiersprache oder Plattform.
    \item Microservices haben einen eigenen Datenhaushalt: eine eigene Datenbank - oder ein vollständig getrenntes Schema in einer gemeinsamen Datenbank.
    \item Microservices können eigene Unterstützungsdienste mitbringen, beispielsweise eine Suchmaschine oder eine spezielle Datenbank. Natürlich gibt es eine gemeinsame Basis für alle Microservices - beispielsweise die Ausführung virtueller Maschinen.
    \item Microservices sind eigenständige Prozesse - oder virtuelle Maschinen, um auch die Unterstützungsdienste mitzubringen.
    \item Dementsprechend müssen Microservices über das Netzwerk kommunizieren. Dazu nutzen Microservices Protokolle, die lose Kopplung unterstützen. Das kann beispielsweise REST sein - oder Messaging-Lösungen.
\end{itemize}
Obwohl der Begriff nicht eindeutig definiert ist, liegt jedoch bei jedem Service-orientierten Systemen, ein Verteiltes System zu Grunde und auch die damit verbundenen Probleme und Herausforderungen.

% Eigene Datenbanken

\section{Gr"o\ss e von Microservices}
\label{sec:groesseMicroservice}
\begin{quotation}
    \frqq Der Name \frqq Microservices\flqq\ verrät schon, dass es um die Servicegröße geht - offensichtlich sollen die Services klein sein."\cite[S. 31]{EWolff2016:Microservices} Es gibt verschiedene Möglichkeiten die Größe von Programmen zu ermitteln. Eine Variante ist zum Beispiel das Zählen von  Lines of Code (LOC), jedoch hat diese Methode auch Nachteile. Denn die Anzahl der Codezeilen hängen stark von der verwendeten Programmiersprache ab. Einige Programmiersprachen benötigen mehr Zeilen Code, um eine bestimmte Tätigkeit abzubilden, als andere.
    \\
    Die Größe von Services sollte jedoch nicht von zentraler Bedeutung sein, denn eine untere Grenze gibt es für Services nicht. "Wohl aber eine obere Grenze: Wenn der Microservice so groß ist, dass er von einem Team nicht mehr weiterentwickelt werden kann, ist sie zu groß. Ein Team sollte dabei eine Größe haben, wie sie für agile Prozesse besonders gut funktioniert. Das sind typischerweise drei bis neun Personen.\flqq\cite[S. 34]{EWolff2016:Microservices}
\end{quotation}

Bei der Größe eines Services ist jedoch darauf zu achten, das ein Service nicht zu viele oder zu wenige Funktionen besitzt. Wie bereits beschrieben, sind Microservices modulare, lose gekoppelte Services. Wird ein Service zu klein angesetzt, können daraus Abhängigkeiten zu anderen Services entstehen und damit das gesetzt der losen Kopplung verletzten. Besitzt hingegen ein Service zu viele Funktionen, wird es meistens nicht mehr als Microservice angesehen, da es nicht eine, sondern mehrere Aufgaben übernimmt und diese wahrscheinlich nicht mehr gut erledigen kann.

\section{Orchestration vs Choreographie}
\label{sec:orchestrationvschoreographie}
Möchte man ein Microservice System aufbauen, stellt sich die Frage, wie einzelne Services Strukturiert werden und wie diese unter einander kommunizieren sollen. Ein bestimmter Vorgang startet in der Regel bei einem Service. Nun muss man entscheiden ob weitere Services hinzugezogen, beziehungsweise informiert werden müssen.
Je nach Anwendungsfall muss man sich zwischen Service Orchestration und Choreographie entscheiden. Dabei ist es fast unmöglich ein ganzes Microservice-System aus nur einem der beiden Varianten zu bauen.

% [Seite 66 EWolff2016] Technologische Wahlfreiheit
\subsection{Herausforderung}
\label{sec:Herausforderung}
Wie bereits in \secref{sec:überblickMicroservice} erwähnt, liegt ein Verteiltes System zu Grunde und damit auch die Grundlegenden Probleme der Kommunikation (siehe \cite[S. 25]{EWolff2016:Microservices}).
Der Ausfall eines Services kann im schlechtesten Fall dazu führen, dass alle anderen Microservices nicht mehr funktionieren. Um das zu verhindern muss klar definiert werden, was Microservices in dieser Situation tun sollen. Zusätzlich muss, sofern Datenbank Operationen eine wichtige Rolle Spielen, das Problem der einheitlichen Transaktion gelöst werden. Wenn zum Beispiel eine Operation Daten über verschiedene Microservices in Datenbanken schreibt, muss bei nicht Erreichbarkeit oder Fehlers eines Microservices eine einheitlicher Rollback durchgeführt werden, um keine inkonsistente Dateien im System zu haben.
Eine weitere Herausforderung besteht in dem Grundkonzept von Microservices. Da nicht definiert ist, welche Programmiersprache für Microservices verwendet wird, kann ein Service zum Beispiel in Java, ein anderes in Scala ode Python geschrieben werden. Es muss daher dafür gesorgt werden, dass die einzelnen Services untereinander interoperabel sind. Um das zu gewährleisten, müssen die Schnittstellen möglichst einheitlich und auf dem gleichen Protokoll aufbauend programmiert werden. Hier bieten REST-Schnittstellen eine gute Lösung. Diese können auf dem HTTP-Protokoll aufgebaut werden. Zusätzlich bietet das HTTP-Protokoll die Möglichkeit, ein einheitliches Medium, wie zum Beispiel XML oder JSON, als Informationsträger zu nutzen. Dabei kann theoretisch jeder Microservice mit jedem anderen Microservice kommunizieren, sofern die Schnittstellen einheitlich definiert sind.

\section{PUSH- VS PULL-Architektur}
\label{sec:PushPullArchitektur}
Grundlegend können Microservices mit Hilfe zwei verschiedener Kommunikations-Architekturen kommunizieren, PUSH- und PULL-Architektur. Dabei ist jedoch nicht ausgeschlossen, dass sobald eine Architektur gewählt worden ist, die andere nicht mehr genutzt werden kann. Genauso wie bei der Entscheidung über die Kommunikationsstruktur (siehe \secref{subsec:orchestration}), kann es von Vorteil sein, beide Architekturen zu nutzen.
\\\\
\textbf{PULL-Architektur}\\
Eine PULL-Architektur basiert auf einen einfachen Request-Replay-Schema. Dementsprechend ist das Web PULL-basiert.
Der Browser macht eine Anfrage an einen Server, dieser wiederum verarbeitet die Anfrage und liefert eine Antwort (Replay) zurück. Dies hat den Vorteil, das nicht lange auf eine Antwort gewartet werden muss und die teilhabenden Kommunikationspartner gegenseitig kennen, jedoch bringt es auch den Nachteil, dass dadurch weitgehend eine synchrone Kommunikation stattfindet und eine Antwort häufig nicht gleichzeitig an mehrere Empfänger senden kann.
\\\\
\textbf{PUSH-Architektur}\\
Eine PUSH-Architektur wird eingesetzt, wenn man verschiedene Kommunikationspartner über bestimmte Ereignisse informieren möchte. Hier stehen meist nicht die Kommunikationspartner, sondern die Informationen im Vordergrund. Dafür wird meistens ein eigenständiger Service (Broadcaster) eingesetzt, der die Verteilung dieser Informationen übernimmt. Dabei kann ein Service als Informationsprovider dienen, zum Beispiel ein Nachrichten-Feed (Von einer Nachrichtenseite). Alle anderen Services abonnieren den Broadcaster und erhalten dadurch alle Nachrichten, die der Informationsprovider sendet.
Es gibt jedoch auch den Fall, dass die Kommunikation sternförmig um den Broadcaster angeordnet sind. Dadurch ist jeder Service der diesen abonniert, sowohl Provider, als auch Consumer.
Anders als bei PULL-basierten Systemen kann hier nicht unbedingt sichergestellt werden, dass alle Nachrichten von allen Konsumern gleichzeitig gelesen und ggf. verarbeitet werden. Jedoch können so Informationen innerhalb eines Microservice-Systems relativ zuverlässig verteilt werden.
Der Vortiel von PUSH-Architekturen ist, dass eine asynchrone Informationsverbreitung aufgebaut werden kann. Zudem können Serviceausfälle, solange es nicht der Boradcaster oder wichtige Microservices sind, überbrückt werden, indem der Broadcaster die Nachrichten für eine bestimmte Zeit vorhält und so der Microservice, welcher nicht erreichbar war, die Nachrichten trotzdem noch erhält.
\\\\
Oft ist es nicht notwendig eine Antwort zu erhalten. Zum Beispiel muss eine Registrierung in unserem fiktiven Unternehmen, der \gmbh\ möglich sein. Dabei sendet der Microservice der für die Registrierung zuständig ist eine einfache Event-Nachricht, wie Benutzer XY hat sich Registriert. Im Hintergrund kann dann zum Beispiel ein anderer Microservice diese Nachricht erhalten und zusätzliche Aktionen durchführen, wie erstellen des Warenkorbs.


% ***************************** BIBLIOGRAPHY **********************************
\baselineskip=14pt
\addcontentsline{toc}{chapter}{\protect\numberline{}\bibname}
\bibliography{bib/thesis}

% ******************************* APPENDIX ************************************
\appendix
\baselineskip=18pt
\chapter{Diagramme und Tabelle}
\label{chap:anhang_a}


% ***************************** BACK MATTER ***********************************
\pagestyle{empty}

\addcontentsline{toc}{chapter}{\protect\numberline{}Eidesstattliche Erklärung}
\chapter*{}
\vspace*{0.5cm}
\noindent
{\bf Eidesstattliche Erklärung} \\
Ich versichere an Eides statt, dass ich die vorliegende Arbeit selbständig
angefertigt und mich keiner fremden Hilfe bedient, sowie keine anderen als die
angegebenen Quellen und Hilfsmittel benutzt habe. Alle Stellen, die wörtlich
oder sinngemäß veröffentlichten oder nicht veröffentlichten Schriften und
anderen Quellen entnommen sind, habe ich als solche kenntlich gemacht. Diese
Arbeit hat in gleicher oder ähnlicher Form noch keiner Prüfungsbehörde
vorgelegen. 

\vspace{1cm}
\toponym, den \today \hfill \author


\vspace*{3cm}
\noindent
{\bf Erklärung} \\
Mir ist bekannt, dass nach § 156 StGB bzw. § 163 StGB eine falsche
Versicherung an Eides Statt bzw. eine fahrlässige falsche Versicherung an
Eides Statt mit Freiheitsstrafe bis zu drei Jahren bzw. bis zu einem Jahr oder
mit Geldstrafe bestraft werden kann. 

\vspace{1cm}
\toponym, den \today \hfill \author


%%%%%%%%%%%%%%%%%%%%%% NICHT IN ARBEIT ÜBERNEHMEN!!! %%%%%%%%%%%%%%%%%%%%%%%%%%
%\newpage
%\noindent
%\vspace*{6cm}
%\begin{center}
%{\bf Spezielle Erklärung vor Beginn der Bachelor-Thesis/Master-Thesis}
%\end{center}
%Hiermit erkläre ich, dass ich die vorausgehenden Seiten, die man sich unter 
%\\[0.2cm]
%{\scriptsize \bf \hspace*{0.3cm}ftp://gatekeeper.informatik.fh-dortmund.de/pub/professors/lenze/thesis/thesis.pdf} \\[0.2cm]
%ansehen kann, mit den Erläuterungen zum Aufbau, zum Umfang und zum Inhalt einer
%Bachelor-/Master-Arbeit sorgfältig 
%gelesen und verstanden habe. Insbesondere ist mir klar, was man unter
%wissenschaftlichem Arbeiten versteht und dass korrektes Zitieren ein
%wesentliches Element in diesem Zusammenhang ist. Alle Fragen, die es in diesem
%Kontext noch gab, habe ich inzwischen mit Herrn Lenze geklärt, und es bestehen
%keine Unklarheiten mehr. Über die besondere Problematik von Plagiaten und den
%Kriterien, die ein Vorliegen anzeigen, bin ich ebenfalls genau unterrichtet.
%
%\vspace{1.5cm}
%\toponym, den \\
%\ \ \  \ \ \hspace*{8cm} {\tiny Unterschrift!} \\
%
%\vfill

\end{document}
