\chapter{Diagramme und Tabelle}
\label{chap:anhang_a}

Ein oder mehrere Anhänge können, müssen aber nicht vorhanden sein.
Als Faustregel gilt: Alles was den Lesefluss stört, kann in einen
Anhang, also insbesondere Programmlistings (länger als eine Seite),
umfangreiches Tabellen-Material, etc.. Die Listings der Programme,
also der {\bf Original-Source-Code}, 
sollten auf jeden Fall -- außer eventuell im Anhang -- auch auf 
CD-ROM  der Bachelor-/Master-Arbeit in einem Einsteckfach beigelegt
werden. Als Zugabe 
kann dort  auch noch direkt ausführbarer Maschinen-Code für verschiedene
Plattformen hinterlegt  werden. Ebenfalls sollte es eine
Selbstinstallationsroutine für mindestens ein gängiges Betriebssystem auf dem
Datenträger geben! \\ 
Im Gegensatz zu normalen Kapiteln werden Anhänge zur besseren Unterscheidung
nicht mit "`1"', "`2"', "`3"',$\ldots$ durchnummeriert, sondern mit
"`A"', "`B"', "`C"',$\ldots$. Ist nur ein Anhang vorhanden, kann die
Nummerierung "`A"' entfallen. \\[0.2cm]
Am Ende des Anhangs sollte der Umfang der Bachelor-/Master-Arbeit etwa 
{\bf 60/100 Seiten} betragen!

\chapter{UML-Diagramme}
\label{chap:anhang_b}

\begin{center}
Hier könnten Klassen- oder UML-Diagramme stehen!
\end{center}

\chapter{Quellcode}
\label{chap:anhang_c}

\begin{center}
Hier könnten konkrete Teile des Java-Quellcodes stehen!
\end{center}